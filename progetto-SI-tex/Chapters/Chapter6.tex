% Chapter Template

\chapter{Caso d'uso: fine-tuning su dataset arbitrario} % Main c

Un caso di studio attuale è quello di prendere una rete allo stato dell'arte, già addestrata per mesi su un dataset enorme come quello di \emph{Imagenet\href{http://www.image-net.org/}} per utilizzarla su un dataset arbitrario a nostra scelta, per scopi industriali o personali.

\label{Capitolo6} % Change X to a consecutive number; for referencing this chapter elsewhere, use \ref{ChapterX}

%--------------------------------------------------------------------
%	SECTION 1
%--------------------------------------------------------------------

\section{Dataset}


%--------------------------------------------------------------------
%	SECTION 2
%--------------------------------------------------------------------

\section{Fine-Tuning}



%--------------------------------------------------------------------
%	SECTION 3
%--------------------------------------------------------------------

\section{ModelZoo}
\parencite{Wzoo}

\subsection{Fine-tuning su Resnet}
Facebook ha reso pubbliche le sue implementazioni di Resnet su github a tutti, cosicché possano essere utilizzate per qualsivoglia esperimento. Ci sono vari modelli, più o meno potenti a seconda degli strati della rete (da 18 a 140). 


\begin{lstlisting}[language={[5.2]Lua}]
--add a Log-SoftMax classifier at the end of the Net
model:add(nn.LogSoftMax())

--criterion (i.e. the loss) will be the negative-likelihood
criterion = nn.ClassNLLCriterion()
\end{lstlisting}
