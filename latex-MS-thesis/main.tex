%%%%%%%%%%%%%%%%%%%%%%%%%%%%%%%%%%%%%%%%%
% Masters/Doctoral Thesis 
% Template license:
% CC BY-NC-SA 3.0 (http://creativecommons.org/licenses/by-nc-sa/3.0/)
%
%%%%%%%%%%%%%%%%%%%%%%%%%%%%%%%%%%%%%%%%%

%----------------------------------------------------------------------------------------
%	PACKAGES AND OTHER DOCUMENT CONFIGURATIONS
%----------------------------------------------------------------------------------------

\documentclass[
11pt, % The default document font size, options: 10pt, 11pt, 12pt
%oneside, % Two side (alternating margins) for binding by default, uncomment to switch to one side
english, % ngerman for German
onehalfspacing, % Single line spacing, alternatives: onehalfspacing or doublespacing
%draft, % Uncomment to enable draft mode (no pictures, no links, overfull hboxes indicated)
%nolistspacing, % If the document is onehalfspacing or doublespacing, uncomment this to set spacing in lists to single
%liststotoc, % Uncomment to add the list of figures/tables/etc to the table of contents
%toctotoc, % Uncomment to add the main table of contents to the table of contents
%parskip, % Uncomment to add space between paragraphs
%nohyperref, % Uncomment to not load the hyperref package
headsepline, % Uncomment to get a line under the header
%chapterinoneline, % Uncomment to place the chapter title next to the number on one line
%consistentlayout, % Uncomment to change the layout of the declaration, abstract and acknowledgements pages to match the default layout
]{MastersDoctoralThesis} % The class file specifying the document structure

\usepackage[utf8]{inputenc} % Required for inputting international characters
\usepackage[T1]{fontenc} % Output font encoding for international characters

%% AGGIUNTI DA ME %%%
%%%%%%%%%%%%%%%%%%%%%
\usepackage{amssymb}
\usepackage{amsmath}
\usepackage{makecell}
\usepackage{subcaption}
\usepackage{booktabs} %beautiful papers tabs
\usepackage{enumerate}

% Use the latin modern font instead of Palatino
\usepackage{lmodern} 
\fontfamily{lmr}\selectfont
%\usepackage{times}
%\fontfamily{ptr}\selectfont

\usepackage{courier}
\usepackage{comment}

%%%% CODE HIGHLITHING %%%%%
%nice way to highlight code 
\usepackage{matlab-prettifier}

\usepackage{listings}
\usepackage{color}
\definecolor{codegreen}{rgb}{0,0.6,0}
\definecolor{codegray}{rgb}{0.5,0.5,0.5}
\definecolor{codepurple}{rgb}{0.58,0,0.82}
\definecolor{backcolour}{rgb}{0.95,0.95,0.92}
\lstdefinestyle{mystyle}{
    backgroundcolor=\color{backcolour},   
    commentstyle=\color{codegreen},
    keywordstyle=\color{magenta},
    numberstyle=\tiny\color{codegray},
    stringstyle=\color{codepurple},
    basicstyle=\footnotesize\ttfamily,
    breakatwhitespace=false,         
    breaklines=true,                 
    captionpos=b,                    
    keepspaces=true,                 
    numbers=left,                    
    numbersep=5pt,                  
    showspaces=false,                
    showstringspaces=false,
    showtabs=false,                  
    tabsize=2,
}
\lstset{style=mystyle}

\usepackage[backend=biber,style=ieee, natbib=true]{biblatex} % Use the bibtex backend with the authoryear citation style (which resembles APA)

\addbibresource{bibliografia.bib} % The filename of the bibliography

\usepackage[autostyle=true]{csquotes} % Required to generate language-dependent quotes in the bibliography

%----------------------------------------------------------------------------------------
%	MARGIN SETTINGS
%----------------------------------------------------------------------------------------

\geometry{
	paper=a4paper, % Change to letterpaper for US letter
	inner=2.5cm, % Inner margin
	outer=3.8cm, % Outer margin
	bindingoffset=.5cm, % Binding offset
	top=1.5cm, % Top margin
	bottom=1.5cm, % Bottom margin
	%showframe, % Uncomment to show how the type block is set on the page
}

%----------------------------------------------------------------------------------------
%	THESIS INFORMATION
%----------------------------------------------------------------------------------------

\thesistitle{Compression of Convolutional Neural Network using Tensor Decomposition} % Your thesis title, this is used in the title and abstract, print it elsewhere with \ttitle
\supervisor{Prof. Stefano \textsc{Mattoccia}} % Your supervisor's name, this is used in the title page, print it elsewhere with \supname
\examiner{} % Your examiner's name, this is not currently used anywhere in the template, print it elsewhere with \examname
\degree{Doctor of Philosophy} % Your degree name, this is used in the title page and abstract, print it elsewhere with \degreename
\author{Ali Alessio Salman} % Your name, this is used in the title page and abstract, print it elsewhere with \authorname
\addresses{} % Your address, this is not currently used anywhere in the template, print it elsewhere with \addressname

\subject{Corso di Laurea Magistrale in Ingegneria Informatica} % Your subject area, this is not currently used anywhere in the template, print it elsewhere with \subjectname
\keywords{} % Keywords for your thesis, this is not currently used anywhere in the template, print it elsewhere with \keywordnames
\university{\href{http://www.university.com}{Alma Mater Studiorum \\University of Bologna}} % Your university's name and URL, this is used in the title page and abstract, print it elsewhere with \univname
\department{\href{http://department.university.com}{DISI \\ Dipartimento di Informatica - Scienza e Ingegneria}} % Your department's name and URL, this is used in the title page and abstract, print it elsewhere with \deptname
%\group{\href{http://researchgroup.university.com}{Research Group Name}} % Your research group's name and URL, this is used in the title page, print it elsewhere with \groupname
%\faculty{\href{http://faculty.university.com}{Faculty Name}} % Your faculty's name and URL, this is used in the title page and abstract, print it elsewhere with \facname

\AtBeginDocument{
\hypersetup{pdftitle=\ttitle} % Set the PDF's title to your title
\hypersetup{pdfauthor=\authorname} % Set the PDF's author to your name
\hypersetup{pdfkeywords=\keywordnames} % Set the PDF's keywords to your keywords
}

\begin{document}

\frontmatter % Use roman page numbering style (i, ii, iii, iv...) for the pre-content pages

\pagestyle{plain} % Default to the plain heading style until the thesis style is called for the body content

%----------------------------------------------------------------------------------------
%	TITLE PAGE
%----------------------------------------------------------------------------------------

%QUESTO DA PROBLEMI CON BIBLATEX SU TEXMAKER
\begin{titlepage}
\begin{center}

\vspace*{.06\textheight}
{\scshape\LARGE \univname\par}\vspace{1.5cm} % University name
\textsc{Scuola di Ingegneria e Architettura \\ \textit{\subjectname} \\[2\baselineskip] \Large Master Thesis }\\[0.5cm] % Thesis type

\HRule \\[0.4cm] % Horizontal line
{\huge \bfseries \ttitle\par}\vspace{0.4cm} % Thesis title
\HRule \\[1.5cm] % Horizontal line

\begin{minipage}[t]{0.4\textwidth}
\begin{flushleft} \large
\emph{Author:}\\
{\authorname} % Author name - remove the \href bracket to remove the link
\end{flushleft}
\end{minipage}
\begin{minipage}[t]{0.4\textwidth}
\begin{flushright} \large
\emph{Thesis Supervisor:} \\
{\supname} \\% Supervisor name - remove the \href bracket to remove the link 
\emph{\\Thesis Advisors}\\
\emph{Ph.D} Matteo Poggi\\ \\
\emph{Ph.D} Fabio Tosi


\end{flushright}
\end{minipage}\\[2cm]

%\vfill
\textit{}\\[0.4cm]
\groupname\\\deptname\\[1cm] % Research group name and department name
 

{\large III Session\\2017/2018}\\[2cm] % Date
%\includegraphics{Logo} % University/department logo - uncomment to place it
 
\vfill
\end{center}
\end{titlepage}

%----------------------------------------------------------------------------------------
%	ABSTRACT PAGE
%----------------------------------------------------------------------------------------
%\begin{comment}
\begin{abstract}
\addchaptertocentry{\abstractname} % Add the abstract to the table of contents

In recent years, deep neural networks have received lots of attention, been applied to a large number of tasks on which they achieved dramatic accuracy improvements. Among these, Convolutional Neural Networks (CNN) have been the key players for visual recognition tasks. These models combine deep networks with the convolution operation, resulting in millions or even billions of parameters and computations, thus, a fundamental ingredient for their success, have been high-end GPUs.\\
However, recent developments in fields such as smart wearable and mobile devices, self-driving cars, virtual reality, unmanned drones among many others, have created an unprecedented opportunity for researchers to tackle fundamental challenges with respect to computer vision. For this reason, it is important to deploy deep learning systems on devices with limited resources for real-time applications. \\
Over the years, the trend for Convolutional Neural Networks saw them grow larger and larger to achieve more impressive performance. For example, the ResNet-50 model\parencite{resnet}, with 50 convolutional layers needs over 95MB memory for storage and much more floating computations to process a single image. Therefore it is crucial, in this context, to investigate new effective ways of both compressing and speeding up these networks. \\
Recent works in literature have shown that it is indeed possible to shrink a model's parameters without sacrificing its accuracy. This is achievable by exploiting their redundant architecture. A convolutional layer can be seen as a four-dimensional tensor with a specific structure. Because of the latter, new works have proposed tensor decomposition methods as a compression for the CNN's convolutional layers. Following this cue, this thesis intends to deepen the knowledge of the aforementioned applications.\\
The thesis is structured as follows: Chapter 1 will introduce the problem stated above and the context in which this thesis will operate. Chapter 2 will illustrate the state of the art techniques regarding CNN compression, while Chapter 3 will provide a thorough analysis of the CNNs architecture. \\
Chapter 4 will then introduce the fundamentals of tensor decomposition and will provide a deep insight into how to apply tensor decomposition on CNNs, ending with a design proposal. Chapter 5 will show, through experiments, how the above-mentioned methods perform on five different models. Finally, Chapter 6 will summarize the work done during the course of this project and will suggest new ideas for future developments. 

\end{abstract}
%\end{comment}


%----------------------------------------------------------------------------------------
%	ABSTRACT PAGE (ITA) 
%----------------------------------------------------------------------------------------
\begin{abstract}
\addchaptertocentry{\abstractname} % Add the abstract to the table of contents

Negli ultimi anni, le reti neurali profonde hanno ricevuto molta attenzione, essendo state applicate in un vasto numero di applicazioni con risultati spesso eccezionali. Tra queste, le reti neurali convoluzionali o Convolutional Neural Networks (CNN), hanno avuto un ruolo chiave per i compiti di riconoscimento visivo. Questi modelli combinano reti neurali profonde con l'operazione di convoluzione, risultando in milioni o addirittura miliardi di parametri e di calcoli; pertanto un ingrediente fondamentale per il loro successo sono state le GPU di fascia alta.
Tuttavia, i recenti sviluppi in settori quali dispositivi indossabili intelligenti e mobili, auto a guida autonoma, realtà virtuale, droni senza pilota e molti altri, hanno creato, per i ricercatori, opportunità senza precedenti di affrontare le sfide fondamentali rispetto alla visione artificiale. Per questo motivo, è importante implementare sistemi di apprendimento profondo (deep learning) su dispositivi con risorse limitate per applicazioni in tempo reale. \\
Ad esempio, il modello ResNet-50 \parencite{resnet}, una rete con 50 strati convoluzionali, ha bisogno di oltre 95 MB di memoria per l'archiviazione e molto di più per i calcoli necessari a classificare una singola immagine. Pertanto, in questo contesto, è fondamentale indagare su nuovi modi efficaci di comprimere e accelerare queste reti. \\
I recenti lavori in letteratura hanno dimostrato che è possibile ridurre i parametri di un modello senza sacrificarne l'accuratezza. Ciò è realizzabile sfruttando la loro architettura ridondante. Uno strato convoluzionale, infatti, può essere visto come un tensore quadridimensionale con una struttura specifica.  Seguendo questo spunto, questa tesi intende approfondire la conoscenza delle applicazioni di cui sopra.
La tesi è strutturata come segue: Il Capitolo 1 introdurrà il problema sopra indicato ed il contesto in cui questa tesi opererà. Il Capitolo 2 illustrerà le tecniche avanzate relative alla compressione delle CNN, mentre il Capitolo 3 fornirà un'analisi approfondita dell'architettura delle CNN. \\
Il Capitolo 4 introdurrà quindi i fondamenti della decomposizione tensoriale e fornirà una visione approfondita di come poterla applicare su una CNN, terminando con una nuova proposta architetturale. Il Capitolo 5 mostrerà, attraverso diversi risultati sperimentali, come i metodi sopra citati si comportino su cinque modelli diversi. Infine, il capitolo 6 riassumerà il lavoro svolto durante il corso di questo progetto lasciando qualche proposta per possibili sviluppi futuri. 

\end{abstract}


%----------------------------------------------------------------------------------------
%	ACKNOWLEDGEMENTS
%----------------------------------------------------------------------------------------

\begin{acknowledgements}
\addchaptertocentry{\acknowledgementname} % Add the acknowledgements to the table of contents
I would like to thank professor Stefano Mattoccia for his patience to supervise this thesis. 
It is always interesting to get in touch with the high-level research of its projects which, again, made me discover new exciting topics while also bringing a precious level of satisfaction to my work. 
In this regard, I wish to thank my two advisors, Matteo Poggi and Fabio Tosi, who happens to be also an exquisite friend. 
\end{acknowledgements}


%----------------------------------------------------------------------------------------
%	LIST OF CONTENTS/FIGURES/TABLES PAGES
%----------------------------------------------------------------------------------------

\tableofcontents % Prints the main table of contents

\listoffigures % Prints the list of figures

\listoftables % Prints the list of tables

%----------------------------------------------------------------------------------------
%	ABBREVIATIONS
%----------------------------------------------------------------------------------------


\begin{abbreviations}{ll} % Include a list of abbreviations (a table of two columns)
\textbf{SVD} &\textbf{S}ingular \textbf{V}alue \textbf{D}ecomposition\\
\textbf{HOSVD} &\textbf{H}igher \textbf{O}rder \textbf{S}ingular \textbf{V}alue \textbf{D}ecomposition\\
\textbf{CPD} & \textbf{C}anonical \textbf{P}olyadic \textbf{D}ecomposition\\
\textbf{SGD} & \textbf{S}tochastic \textbf{G}radient \textbf{D}escent\\
\textbf{NN} &\textbf{N}eural \textbf{N}etwork\\
\textbf{CNN} & \textbf{C}onvolutional \textbf{N}eural \textbf{N}etwork\\
\textbf{CONV} &\textbf{Conv}olutional layer \\
\textbf{BN} &\textbf{B}atch \textbf{N}ormaization\\
\textbf{TD block} &\textbf{T}ensor \textbf{D}ecomposition block\\

\end{abbreviations}


%----------------------------------------------------------------------------------------
%	SYMBOLS
%----------------------------------------------------------------------------------------
\begin{comment}
\begin{symbols}{lll} % Include a list of Symbols (a three column table)

$a$ & distance & \si{\meter} \\
$P$ & power & \si{\watt} (\si{\joule\per\second}) \\
%Symbol & Name & Unit \\

\addlinespace % Gap to separate the Roman symbols from the Greek

$\omega$ & angular frequency & \si{\radian} \\

\end{symbols}
\end{comment}

%----------------------------------------------------------------------------------------
%	DEDICATION
%----------------------------------------------------------------------------------------

\dedicatory{
A mio padre, che mi ha insegnato a giocare\\
A mia zia, che mi ha insegnato la pazienza\\
A mio nonno, che mi ha insegnato a non prendermi sul serio\\
A quel cuore che oggi, nonostante tutto, saprebbe battere più forte del mio.\\


} 


%----------------------------------------------------------------------------------------
%	THESIS CONTENT - CHAPTERS
%----------------------------------------------------------------------------------------

\mainmatter % Begin numeric (1,2,3...) page numbering

\pagestyle{thesis} % Return the page headers back to the "thesis" style

% Include the chapters of the thesis as separate files from the Chapters folder
% Uncomment the lines as you write the chapters

\chapter{Reti Neurali Artificiali: le basi} % Main chapter title

\label{Capitolo1} % Change X to a consecutive number; for referencing this chapter elsewhere, use \ref{ChapterX}
%variables to define path to images
\def \path {Figures/C1}
\def \teoria {Figures/teoria}

%--------------------------------------------------------------------%--------------------
%	SECTION 1
%--------------------
%--------------------------------------------------------------------

\section{Breve introduzione}
\label{sec:intro}
%COSA SONO LE RETI NEURALI ARTIFICIALI
Una rete neurale artificiale – chiamata normalmente solo rete neurale (in inglese \emph{Neural Network}) – è
un modello di calcolo adattivo, ispirato ai principi di funzionamento del sistema nervoso degli organismi evoluti che secondo l'approccio connessionista\parencite{WConnessionismo} possiede una complessità non descrivibile con i metodi simbolici.  
La caratteristica fondamentale di una rete neurale è che essa è capace di acquisire conoscenza modificando la propria struttura in base alle informazioni esterne (i dati in ingresso) e interne (tramite le connessioni) durante il processo di apprendimento. Le informazioni sono immagazzinate nei parametri della rete, in particolare, nei pesi associati alle connessioni. 
Sono strutture non lineari in grado di simulare relazioni complesse tra
ingressi e uscite che altre funzioni analitiche non sarebbero in grado di fare. 


L'unità base di questa rete è il neurone artificiale introdotto per la prima volta da McCulloch e
Pitts nel 1943 (fig. \ref{fig:neuron}).


\begin{figure}[h!]
 \centering
 \includegraphics[width=1.0\textwidth]{\teoria/NeuronePitts.png}
 \caption{Modello di calcolo di un neurone (a sinistra) e schema del neurone artificiale (a destra)}
 \label{fig:neuron}
\end{figure}

Si tratta di un'unità di calcolo a N ingressi e 1 uscita. Come si può vedere dall'immagine a
sinistra gli ingressi rappresentano le terminazioni sinaptiche, quindi sono le uscite di altrettanti
neuroni artificiali. A ogni ingresso corrisponde un peso sinaptico $w$, che stabilisce quanto quel
collegamento sinaptico influisca sull'uscita del neurone. Si determina quindi il potenziale del neurone facendo una somma degli ingressi, pesata secondo i pesi $w$. \\
A questa viene applicata una funziona di trasferimento non lineare: 
\begin{equation}
 f(x) = H(\sum_{i}(w_i x_i))
\end{equation} 
ove $H$ è la funzione gradino di Heaviside \parencite{WHeaviside}. Vi sono, come vedremo, diverse altre funzioni non lineari tipicamente utilizzate come funzioni di attivazioni dei neuroni. 
Nel '58 Rosenblatt propone il modello di \emph{Percettrone} rifinendo il modello di neurone a soglia, aggiungendo un termine di \emph{bias} e un algoritmo di apprendimento basato sulla minimizzazione dell'errore, cosiddetto \emph{error back-propagation}\parencite{WPercettrone}.
\begin{equation}
 f(x) = H(\sum_{i}(w_i x_i)+b),\quad ove \quad b = bias \\
\end{equation}
\begin{equation} 
 w_i(t+1) = w_i(t)+\eta \delta x_i(t)
\end{equation}
dove $\eta$ è una costante di apprendimento strettamente positiva che regola la velocità di apprendimento, detta \emph{learning rate} e $\delta$ è la discrepanza tra l'output desiderato e l'effettivo output della rete. 

Il percettrone però era in grado di imparare solo funzioni linearmente separabili. Una maniera per oltrepassare questo limite è di combinare insieme le risposte di più percettroni, secondo architetture multistrato. 
%--------------------------------------------------------------------%--------------------
%	SECTION 2
%--------------------
%--------------------------------------------------------------------

\section{Multi-layer Perceptron}

Il Multi-layer Perceptron (\textit{MLP}) o percettrone multi-strato è un tipo di rete feed-forward che mappa un set di input ad un set di output. È la naturale estensione del percettrone singolo e permette di distinguere dati non linearmente separabili.

\begin{figure}[h!]
 \centering
 \includegraphics[width=1.0\textwidth]{\teoria/multilayer.png}
 \caption{Struttura di un percettrone multistrato con un solo strato nascosto}
 \label{fig:multilayer}
\end{figure}


Il \emph{mlp} possiede le seguenti caratteristiche: 
\begin{itemize}
\item ogni neurone è un percettrone come quello descritto nella sezione \ref{sec:intro}. Ogni unità possiede quindi una propria funzione d'attivazione non lineare.
\item a ogni connessione tra due neuroni corrisponde un peso sinaptico $w$
\item è formato da 3 o più strati. In \ref{fig:multilayer} è mostrato un MLP con uno strato di input; un solo strato nascosto (o \emph{hidden layer}; ed uno di output.) 
\item l'uscita di ogni neurone dello strato precedente è l'ingresso per ogni neurone dello
strato successivo. È quindi una rete \emph{completamente connessa}. Tuttavia, si possono
disconnettere selettivamente settando il peso sinaptico $w$ a 0.
\item la dimensione dell'input e la dimensione dell'output dipendono dal numero di
neuroni di questi due strati. Il numero di neuroni dello strato nascosto è invece
indipendente, anche se influenza di molto le capacità di apprendimento della rete. 
\end{itemize}
%non linearità 
Se ogni neurone utilizzasse una funzione lineare allora si potrebbe ridurre l'intera rete ad una composizione di funzioni lineari. Per questo - come detto prima - ogni neurone possiede una funzione di attivazione non lineare. 

%black box
\subsection{Strati Nascosti}
I cosiddetti \emph{hidden layers} sono una parte molto interessante della rete. Per il teorema di
approssimazione universale\parencite{WApprox}, \cite{WApprox} \parencite{WConnessionismo} una rete con un singolo strato nascosto e un numero finiti di
neuroni, può essere addestrata per approssimare una qualsiasi funzione continua su uno spazio compatto di $\mathbb{R}^n$. In altre parole, un singolo strato nascosto è abbastanza potente da imparare un ampio numero di funzioni. Precisamente, una rete a 3 strati è in grado di separare regioni convesse con un numero di lati $\leqslant$ numero neuroni nascosti. 

Reti con un numero di strati nascosti maggiore di 3 vengono chiamate reti neurali profonde o \emph{deep neural network}; esse sono in grado di separare regioni qualsiasi, quindi di approssimare praticamente qualsiasi funzione. Il primo e l’ultimo strato devono avere un numero di neuroni pari alla dimensione dello spazio di ingresso e quello di uscita. Queste sono le terminazioni della \emph{"black box"} che rappresenta la funzione che vogliamo approssimare. 

L'aggiunta di ulteriori strati non cambia \emph{formalmente} il numero di funzioni che si possono approssimare; tuttavia vedremo che nella pratica un numero elevato di strati migliori di gran lunga le performance della rete su determinati task, essendo gli hidden layers gli strati dove la rete memorizza la propria rappresentazione astratta dei dati in ingresso. Nel capitolo 4 vedremo un'architettura all'avanguardia con addirittura 152 strati.
 
%--------------------------------------------------------------------%--------------------
%	SECTION 3
%--------------------
%--------------------------------------------------------------------

\section{Caso di studio: prevedere il profitto di un ristorante}
%%%  30 Coperti - 11 Mesi all'anno
%%%  20 coperti - 38-42 ore di apertura a settimana 
%%%  Profitto in % = Ricavo - Spesa, According to the Restaurant Resource Group, average profit margins for restaurants range from 2 to 6 percent. 
%%% 
Prendendo spunto dalla traccia d'esame di Sistemi Intelligenti M del 2 Aprile 2009: \\
\textit{"Loris è figlio della titolare di una famoso spaccio di piadine nel Riminese e sta tornando in Italia
dopo aver frequentato con successo un prestigioso Master in Business Administration ad Harvard, a
cui si è iscritto inseguendo il sogno di esportare in tutto il mondo la piadina romagnola. Nel lungo
viaggio in prima classe, medita su come presentare alla mamma, che sa essere un tantino restia alle
innovazioni, il progetto di aprire un ristorantino a New York City."}

Loris ha esportato con successo la piadina a NY, si veda \parencite{WGradisca} \href{www.gradiscanyc.com} ma col passare degli anni ha notato alcuni problemi e vuole utilizzare di nuovo le sue brillanti capacità analitiche per migliorare il profitto del suo ambizioso ristorante. \\
I problemi sono 2: 
\begin{enumerate}
\item il numero di coperti del ristorante è troppo grande rispetto alla clientela che effettivamente viene;
\item gli orari di apertura sono troppo lunghi e vi sono alcune zone morte dove il costo di mantenere aperto il ristorante è maggiore rispetto al ricavo dei pochi clienti che mangiano a quell'ora; 
\end{enumerate}

%--------------------------------------------------------------------%--------------------
%	SECTION 4
%--------------------
%--------------------------------------------------------------------

\section{Implementazione da zero di un MLP}
%Parlare brevemente del caso d'uso 
\subsection{Dataset e Modello iniziale}

\subsection{Forward Propagation}
\subsection{Backpropagation}
Come si fa ad addestrare una rete multistrato con diversi neuroni per strato? Tramite l'algoritmo di \emph{backpropagation of errors} di ... 
\subsection{Verifica numerica del gradiente}
\subsection{Addestramento}

%--------------------------------------------------------------------%--------------------
%	SECTION 5
%--------------------
%--------------------------------------------------------------------

\section{Ottimizzazione: diverse tecniche}
%SGD ASGD LBGFS ADAM 
%INSERIRE GRAFICI 


%--------------------------------------------------------------------%--------------------
%	SECTION 6
%--------------------
%--------------------------------------------------------------------
\section{Overfitting: come rilevarlo e risolverlo}



%--------------------------------------------------------------------%--------------------
%	SECTION 7
%--------------------
%--------------------------------------------------------------------

\section{Risultati}

\chapter{Implementazione di un MLP} % Main chapter title
\label{Capitolo2} % Change X to a consecutive number; for referencing this chapter elsewhere, use \ref{ChapterX}
%variables to define path to images
\def \path {Figures/C1}
\def \teoria {Figures/teoria}
%Parlare brevemente del caso di studio
Per il caso di studio introdotto nel Capitolo \ref{Capitolo1} si è implementato da zero un percettrone multistrato a 3 livelli come quello illustrato in sezione \ref{sec:mlp}. Vediamo qui di seguito le varie parti da implementare passo passo per costruire un MLP, addestrarlo e verificare che l'addestramento sia stato eseguito in maniera corretta. Il progetto è realizzato in Lua, per utilizzare il framework per il \emph{Machine Learning} \textbf{Torch}(si veda l'appendice \ref{AppendixB}) e mantenere le consistenza con i capitoli successivi, nei quali si userà nuovamente Torch per addestrare reti neurali molto più complesse.

%--------------------------------------------------------------------
%	SECTION 1
%--------------------------------------------------------------------
\section{Dataset e Architettura}
Nei vari anni, Loris ha cambiato le due variabili in gioco annotando di volta in volta i risultati. Siccome i coperti erano troppo pochi e le file d'attesa erano troppo lunghe, il ristorante perdeva alcuni clienti. Quindi Loris ha portato i coperti a 25 diminuendo le ore settimanali a 38, ed i profitti sono aumentati. Tuttavia, non era raro che ancora qualche cliente dovesse aspettare in piedi per troppo tempo (si sa la vita a NY è frenetica), finendo poi per scegliere un ristorante adiacente. Inoltre, aveva diminuito troppo drasticamente le ore; nel weekend i clienti arrivavano fino a tardi, quindi scegliere di rimanere aperti un'ora in più sarebbe stato lungimirante. Così, dopo l'allargamento della sala principale, ha aggiunto altri coperti ed aumentato le ore settimanali a 40, segnando un record personale di $4.4\%$ di profitti annui. \\

Quindi, i dati in ingresso ed in uscita, in $X$ e $Y$ rispettivamente sono:
\[
X = \begin{pmatrix}
22 & 42\\
25 & 38 \\
30 & 40
\end{pmatrix}
%
Y = \begin{pmatrix}
2.8\\
3.4 \\
4.4
\end{pmatrix}
\]
Osservando le dimensioni dei dati si nota che la rete deve avere 2 input e dare in uscita 1 output, che chiameremo $\hat{y}$, in contrapposizione a $y$ che è l'uscita desiderata. Per quanto detto nella sezione \ref{sec:mlp}, il MLP deve avere 2 neuroni nello strato di ingresso ed 1 solo in uscita. Inoltre, avrà uno strato nascosto con 3 neuroni. La dimensione di ogni strato fa parte di un insieme di parametri che viene deciso "a mano" sperimentando, i cosiddetti \emph{hyperparameters}. Questi parametri non vengono aggiornati durante l'addestramento - come i pesi della rete - ma vengono decisi a priori. \\In figura \ref{fig:mlp} è mostrata l'architettura generale della nostra rete. \newpage
\begin{figure}[h!]
 \centering
 \includegraphics[width=1.0\textwidth]{\path/MLP-Profitto.png}
 \caption{Architettura del MLP per la previsione dei profitti}
 \label{fig:mlp}
\end{figure}

Di seguito gli snippet di codice per la definizione dei dati e dell'architettura della rete secondo lo schema appena presentato.
\begin{lstlisting}[language={[5.2]Lua}]
----------------------- Part 1 ----------------------------
th = require 'torch'
bestProfit = 6.0
-- X = (num coperti, ore di apertura settimanali), y = profitto lordo annuo in percentuale
torch.setdefaulttensortype('torch.DoubleTensor')
X = th.Tensor({{22,42}, {25,38}, {30,40}})
y = th.Tensor({{2.8},{3.4},{4.4}})

--normalize
normalizeTensorAlongCols(X)
y = y/bestProfit

\end{lstlisting}

\begin{lstlisting}[language={[5.2]Lua}]
----------------------- Part 2 ----------------------------
--creating the NN class in Lua, using a nice class utility
class=require 'class'
local Neural_Network = class('Neural_Network')

function Neural_Network:__init(inputs, hiddens, outputs)
      self.inputLayerSize = inputs
      self.hiddenLayerSize = hiddens
      self.outputLayerSize = outputs
      self.W1 = th.randn(net.inputLayerSize, self.hiddenLayerSize)
      self.W2 = th.randn(net.hiddenLayerSize, self.outputLayerSize)
end
\end{lstlisting}

\section{Forward Propagation}

%%% MULTI-COLUMN table for variables
\begin{table}[h!]
\caption{Variabili usate nel testo e nel codice}
\label{tab:variabili}
\begin{center}
\begin{tabular}{ |p{3cm}||p{3cm}|p{3cm}|p{3cm}|  }
 \hline
 \multicolumn{4}{|c|}{\textbf{Variabili}} \\
 \hline
 \textbf{S. Codice} & \textbf{S. Matematico} & \textbf{Definizione} & \textbf{Dimensione}\\
 \hline
 X	& $X$	&Esempi, 1 per riga&	\makecell{(numEsempi, \\inputLayerSize)}\\
 \hline
 y&	$y$	& uscita desiderata	& \makecell{(numEsempi, \\outputLayerSize)}\\
 \hline
 W1 & $W^{(1)}$	& Pesi layer 1&	\makecell{(inputLayerSize, \\hiddenLayerSize)}\\
 \hline
 W2	& $W^{(2)}$   & Pesi layer 2&	(hiddenLayerSize, outputLayerSize)\\
 \hline
 z2 & $z^{(2)}$	& Input layer 2& \makecell{(numEsempi, \\hiddenLayerSize)}\\
 \hline
 a2 & $a^{(2)}$	& Uscita layer 2& \makecell{(numEsempi, \\hiddenLayerSize)}\\
 \hline
 z3 & $z^{(3)}$	& Input layer 3& \makecell{(numEsempi, \\outputLayerSize)}\\
  \hline
 \end{tabular}
 \end{center}
 \end{table}
Nella tabella \ref{tab:variabili} sono elencate le variabili della rete. Gli input dei layer indicati con $z$ possono anche essere chiamati "attività dei layer" (indicando l'attività sulle loro sinapsi); e $a^{(2)}$ indica l'uscita del neurone dopo aver applicato la sommatoria e la funzione di attivazione sulle attività provenienti dal layer precedente.

Per muovere i dati in parallelo attraverso la rete si usa la moltiplicazione fra matrici, per questo è molto comodo usare framework che supportano operazioni fra matrici come \emph{Torch, Numpy o Matlab}. Per prima cosa, gli input del tensore $X$ devono essere moltiplicati e sommati con i pesi del primo layer $W^{(1)}$, ottenendo l'ingresso per l'hidden layer:
\begin{equation}
z^{(2)} = XW^{(1)} \tag{1}
\end{equation}
Si noti che $z^{(2)}$ è di dimensione 3x3, essendo $X$ e $W^{(1)}$ di dimensione 3x2 e 2x3 rispettivamente. \\
Ora bisogna applicare la funzione di attivazione a $z^{(2)}$. Vi sono diverse funzioni di attivazione utilizzate per le reti neurali. Una delle prime a diventare popolare fu la funzione \emph{sigmoide} \parencite{WSigmoid}, utilizzata per questa rete. Vedremo nei capitoli successivi funzioni più efficaci.
\begin{equation}
a^{(2)} = f(z^{(2)}) \tag{2},\quad ove \quad f=sigmoide
\end{equation}
Per completare la \emph{forward propagation}, bisogna seguire lo stesso procedimento per lo strato di output: sommare i contributi provenienti dall'hidden layer ed applicare la sigmoide:
\begin{center}
\begin{align*}
z^{(3)} = a^{(2)}W^{(2)} \tag{3}\\
\hat{y} = f(z^{(3)}) \tag{4}
\end{align*}
\end{center}
Essendo $a^{(2)}$ di dimensione 3x3 e $W^{(2)}$ 3x1 l'output $\hat{y}$ sarà anch'esso di dimensione 3x1, risultando quindi in una previsione per ogni esempio in ingresso. \\
Si noti come la moltiplicazione fra matrici renda tutto esprimibile in poche righe di codice.
\begin{lstlisting}[language={[5.2]Lua}]
--Note: I didn't implement manually the sigmoid function as Torch has one built-in.
--define a forward method
function Neural_Network:forward(X)
   --Propagate inputs though network
   self.z2 = th.mm(X, self.W1) --matrix multiplication
   self.a2 = th.sigmoid(self.z2)
   self.z3 = th.mm(self.a2, self.W2)
   yHat = th.sigmoid(self.z3)
   return yHat
end
\end{lstlisting}
\section{Backpropagation}
\label{sec:backprop}
Addestrare una rete multistrato con diversi neuroni per strato, ognuno dei quali con uscita non lineare, non è semplice. Fortunatamente, Rumelhart-Hinton-Williams nel 1985 idearono l'algoritmo che è tutt'ora alla base dell'apprendimento delle reti neurali, la \emph{backpropagation of errors}.
Non si può introdurre l'algoritmo di \emph{"backprop"} senza prima spiegare il concetto di funzione di costo (o \emph{loss function}). Nelle reti neurali (e più specificatamente nell'apprendimento supervisionato \parencite{WSupervised}, si veda anche sezione \ref{subsec:supervised}), la funzione di costo misura la discrepanza tra l'uscita desiderata e l'effettivo output della rete. È quindi una misura dell'errore della rete, per cui l'obiettivo dell'apprendimento è trovare il minimo di questa funzione (modificando la struttura interna della rete, ovvero i pesi sinaptici).
Come per la funzione di attivazione, anche in questo caso ci sono ampie possibilità di scelta \parencite{WLoss} a seconda del task su cui la rete viene addestrata. \\
E di nuovo, come per la funzione di attivazione, si è scelta una delle funzione di costo più popolari: l'errore quadratico medio.
\begin{equation}
J = \sum_{j=1}^{n} \frac{1}{2} (y-\hat{y})^2} \tag{5}
\end{equation}
Da cui, il codice in Lua:
\begin{lstlisting}[language={[5.2]Lua}]
function Neural_Network:costFunction(X, y)
   --Compute the cost for given X,y, use weights already stored in class
   self.yHat = self:forward(X)
   J = 0.5 * th.sum(th.pow((y-yHat),2))
   return J
end
\end{lstlisting}

La \emph{backprop} ha alcuni requisiti:
\begin{itemize}
\item Reti stratificate;
\item Ingressi a valori reali $\in [0,1]$;
\item Neuroni non lineari con funzione di uscita sigmoidale (o altra fz. di attivazione derivabile).
\end{itemize}
Sotto queste condizioni l'algoritmo sfrutta la regola della catena \parencite{WChain} per la derivazione di funzione composte, per calcolare il gradiente della \emph{funzione di costo}. I pesi della rete vengono quindi aggiornati secondo la \emph{discesa del gradiente} (figura \ref{fig:gradDescend}); ovvero variano in maniera tale da minimizzare la funzione di costo $J$.
\begin{figure}[h!]
 \centering
 \includegraphics[width=1.0\textwidth]{\teoria/gradientbased.png}
 \caption{Cercare il minimo di una fz. seguendo la discesa del gradiente}
 \label{fig:gradDescend}
\end{figure}

Viene chiamato \emph{backward propagation of errors} poiché l'errore calcolato a partire dall'output della rete viene distribuito in maniera proporzionale all'indietro, su tutti i neuroni della rete. È importante quindi, spezzare il calcolo del gradiente dell'errore in derivate parziali, dall'ultimo strato fino al primo, e poi combinarle insieme.

Si noti che le equazioni (1-5) formano una un'unica equazione che lega $J$ a $X, y, W^{(1)}, W^{(2)}$. Tenendo questo in mente, si applica la regola della catena. \\
%% mettere la formula completa da spezzare in passaggi %%
Partendo dal fattore riguardante lo strato di output si ha:
$$
\frac{\partial J}{\partial W^{(2)}} = \sum \frac{\partial \frac{1}{2}(y-\hat{y})^2}{\partial W^{(2)}}
$$
Sviluppando i calcoli si ottiene:
$$
\frac{\partial J}{\partial W^{(2)}} = -(y-\hat{y}) \frac{\partial \hat{y}}{\partial W^{(2)}}
$$
L'equazione (4) indica che $\hat{y}$ è la funzione di attivazione di $z^{(3)}$. Da cui:
$$
\frac{\partial J}{\partial W^{(2)}} =
-(y-\hat{y})
\frac{\partial \hat{y}}{\partial z^{(3)}}
\frac{\partial z^{(3)}}{\partial W^{(2)}}
$$
Il 2° membro dell'equazione è semplicemente la derivata della funzione di attivazione sigmoide (fig. \ref{fig:sigmoidPrime}):
\begin{align*}
f(z) = \frac{1}{1+e^{-z}}\\
f^\prime(z) = \frac{e^{-z}}{(1+e^{-z})^2}
\end{align*}
Definiamola, quindi, nel codice:
%% qui codice %%
\begin{lstlisting}[language={[5.2]Lua}]
function Neural_Network:d_Sigmoid(z)
   --Derivative of sigmoid function
   return th.exp(-z):cdiv( (th.pow( (1+th.exp(-z)),2) ) )
end
\end{lstlisting}
%% qui grafico uguale al iPython %%
\begin{figure}[h!]
 \centering
 \includegraphics[width=0.5\textwidth]{\path/sigmoidPrime.png}
 \caption{La funzione sigmoide e la sua derivata}
 \label{fig:sigmoidPrime}
\end{figure}

L'equazione così ottenuta è:
$$
\frac{\partial J}{\partial W^{(2)}}=
-(y-\hat{y}) f^\prime(z^{(3)}) \frac{\partial z^{(3)}}{\partial W^{(2)}}
$$

Infine, dobbiamo trovare $\frac{\partial z^{(3)}}{\partial W^{(2)}}$, che rappresenta la variazione dell'attività del terzo layer rispetto ai pesi del secondo layer. Richiamando l'equazione (3):
$$
z^{(3)} = a^{(2)}W^{(2)} \tag{3}\\
$$
Tralasciando per un attimo la somma tra i vari neuroni, si nota una semplice relazione lineare fra i termini, con a2 che rappresenta la pendenza. Indi:
$$
\frac{\partial z^{(3)}}{\partial W^{(2)}} = a^{(2)}
$$
Indicando con $\delta^{(3)}$, l'errore sullo strato di uscita, si ha:
$$
\delta^{(3)} = -(y-\hat{y}) f^\prime(z^{(3)})
$$
Ora bisogna moltiplicare l'errore con $a^{(2)}$. Come indicato nelle ottime dispense di CS231 di Stanford \parencite{WCS231vec}: guardare alle dimensioni delle matrici può essere utile in questo caso. Infatti per fare combaciare le dimensioni, c'è solo una maniera di calcolare la derivata qui, ed è facendo la trasposta di $a^{(2)}$:
$$
\frac{\partial J}{\partial W^{(2)}} =
(a^{(2)})^T\delta^{(3)}\tag{6}
$$
Si noti che la sommatoria che abbiamo tralasciato all'inizio del calcolo viene inclusa "automaticamente" dalle somme delle moltiplicazione fra matrici.
\\

L'altro termine da calcolare è $\frac{\partial J}{\partial W^{(1)}}$.
Il calcolo è inizialmente simile a quello precedente, iniziando sempre dalla derivata sull'ultimo strato ed utilizzando i risultati trovati in precedenza:
$$
\frac{\partial J}{\partial W^{(1)}} = (y-\hat{y})
\frac{\partial \hat{y}}{\partial W^{(1)}}
$$

$$
\frac{\partial J}{\partial W^{(1)}} = (y-\hat{y})
\frac{\partial \hat{y}}{\partial z^{(3)}}
\frac{\partial z^{(3)}}{\partial W^{(1)}}
$$

$$
\frac{\partial J}{\partial W^{(1)}} = -(y-\hat{y}) f^\prime(z^{(3)}) \frac{\partial z^{(3)}}{\partial W^{(1)}}
$$
$$
\frac{\partial J}{\partial W^{(1)}} = \delta^{(3)} \frac{\partial z^{(3)}}{\partial W^{(1)}}
$$
\\
\\
Ora rimane l'ultimo termine da calcolare, anch'esso da scomporre in diversi fattori andando a ritroso nella rete:
$$
\frac{\partial z^{(3)}}{\partial W^{(1)}} = \frac{\partial z^{(3)}}{\partial a^{(2)}}\frac{\partial a^{(2)}}{\partial W^{(1)}}
$$
Come prima, c'è una relazione lineare tra le sinapsi, ma stavolta la pendenza è data da $W^{(2)}$; anche in questo caso da trasporre.
$$
\frac{\partial J}{\partial W^{(1)}} = \delta^{(3)}
(W^{(2)})^{T}
\frac{\partial a^{(2)}}{\partial W^{(1)}}
$$
$$
\frac{\partial J}{\partial W^{(1)}} = \delta^{(3)}
(W^{(2)})^{T}
\frac{\partial a^{(2)}}{\partial z^{(2)}}
\frac{\partial z^{(2)}}{\partial W^{(1)}}
$$
$\frac{\partial a^{(2)}}{\partial z^{(2)}}$ è di nuovo la derivata della $f$ di attivazione. Il termine finale del calcolo $\frac{\partial z^{(2)}}{\partial W^{(1)}}$, rappresenta quanto varia l'uscita del primo strato al variare dei pesi. Richiamando l'equazione (1) si nota subito che questo valore è dato dal vettore di input $X$ - come prima - traposto:
$$
\frac{\partial J}{\partial W^{(1)}} =
X^{T}
\delta^{(3)}
(W^{(2)})^{T}
f^\prime(z^{(2)})
$$
Chiamando $\delta^{(2)} = \delta^{(3)} (W^{(2)})^{T} f^\prime(z^{(2)})$ diventa:
$$
\frac{\partial J}{\partial W^{(1)}} =
X^{T}\delta^{(2)} \tag{7}
$$

Facendo un sommario:

\begin{equation}
\boxed{\frac{\partial J}{\partial W^{(2)}} =
(a^{(2)})^T\delta^{(3)}\tag{6}}
\end{equation}
\begin{equation}
\boxed{\frac{\partial J}{\partial W^{(1)}} =
X^{T}\delta^{(2)} \tag{7}}
\end{equation}
\begin{equation}
\boxed{\delta^{(2)} = \delta^{(3)} (W^{(2)})^{T} f^\prime(z^{(2)}) \tag{8}}
\end{equation}
\begin{equation}
\boxed{\delta^{(3)} = -(y-\hat{y}) f^\prime(z^{(3)})  \tag{9}} \end{equation}
\\
Implementando le equazioni sovrascritte in Lua, la classe \texttt{Neural\_Network} è quindi completa (per i dettagli si veda l'appendice \ref{AppendixA}).

\begin{lstlisting}[language={[5.2]Lua}]
function Neural_Network:d_CostFunction(X, y)
   --Compute derivative wrt to W1 and W2 for a given X and y
   self.yHat = self:forward(X)
   delta3 = th.cmul(-(y-self.yHat), self:d_Sigmoid(self.z3))
   dJdW2 = th.mm(self.a2:t(), delta3)

   delta2 = th.mm(delta3, self.W2:t()):cmul(self:d_Sigmoid(self.z2))
   dJdW1 = th.mm(X:t(), delta2)

   return dJdW1, dJdW2
end
\end{lstlisting}

\section{Verifica numerica del gradiente}
\label{sec:gradcheck}
Siccome la backprop è notoriamente difficile da debuggare una volta che la si usa per l'addestramento di una rete, bisogna controllare se l'implementazione della sezione precedente è corretta prima di proseguire nel progetto. A questo scopo, è stata scritta una funzione per il calcolo \emph{numerico} del gradiente che andrà poi confrontata con il calcolo computato dalla Backprop.
\\	
L'algoritmo \parencite{WGradcheck} è basato sulla seguente definizione di derivata:
$$
\frac{\mathrm{d} }{\mathrm{d} \theta}J(\theta)) = \lim_{\epsilon\rightarrow 0} \frac{J(\theta + \epsilon)- J(\theta - \epsilon)}{2*\epsilon}
$$
Il gradiente che bisogna controllare è formato dai 2 vettori che contengono le derivate dei pesi di tutta la rete: $\frac{\partial J}{\partial W^{(1)}} \quad e \quad \frac{\partial J}{\partial W^{(2)}}}$.\\
Approssimando $\epsilon$ con un valore molto piccolo \emph{(i.e. $10^{-4}$)} è possibile perturbare singolarmente i pesi della rete (contenuti nei vettori $W^{(1)}$ e $W^{(2)}$) e calcolare così il gradiente in maniera numerica.

Per poterlo fare, servono delle funzioni ausiliare (metodi getter e setter) per prendere e settare i pesi ed il gradiente della rete come singolo vettore "flat" di parametri (appendice \ref{AppendixA}). Dopodiché basta un loop ed un array per memorizzare le derivate dei singoli pesi così calcolati:

\begin{lstlisting}[language={[5.2]Lua}]
function computeNumericalGradient(NN, X, y)
   paramsInitial = NN:getParams()
   numgrad = th.zeros(paramsInitial:size())
   perturb = th.zeros(paramsInitial:size())
   e = 1e-4

   for p=1,paramsInitial:nElement() do
      --Set perturbation vector
      perturb[p] = e
      NN:setParams(paramsInitial + perturb)
      loss2 = NN:costFunction(X, y)

      NN:setParams(paramsInitial - perturb)
      loss1 = NN:costFunction(X, y)

      --Compute Numerical Gradient
      numgrad[p] = (loss2 - loss1) / (2*e)

      --Return the value we changed to zero:
      perturb[p] = 0
   end

   --Return Params to original value:
   NN:setParams(paramsInitial)
   return numgrad
end
\end{lstlisting}
Si può ora inizializzare una rete, eseguire la backpropagation e confrontare i valori con quelli calcolati numericamente:
\begin{lstlisting}[language={[5.2]Lua}]
--test if we actually make the calculations correctly
NN = Neural_Network(2,3,1)

print('Gradient checking...')
numgrad = computeNumericalGradient(NN, X, y)
grad = NN:computeGradients(X, y)
--[[
In order to make an accurate comparison of the 2 vectors
we can calculate the difference as the ratio of:
numerator  --> the norm of the difference
denumerator--> the norm of the sum
Should be in the order of 10^-8 or less
--]]
diff = th.norm(grad-numgrad)/th.norm(grad+numgrad)
print(string.format('The difference is %e',diff))
\end{lstlisting}

Per calcolare \emph{quanto} siano effettivamente uguali i due gradienti si può usare un rapporto basato sulla norme della somma e della differenza dei gradienti (si veda il codice sopra). Se la backprop è stata implementata correttamente questa differenza dovrebbe essere nell'ordine di $10^{-8}$ o inferiore. Difatti, quando si esegue lo script si ottiene:


%%inserire figura %%
\begin{lstlisting}
$ th 4_gradCheck.lua 
Gradient checking...	
The difference is 2.123898e-10	
\end{lstlisting}

\section{Addestramento}
\label{ref:training}
%--------------------------------------------------------------------%--------------------
%	SECTION 5
%--------------------
%--------------------------------------------------------------------
Una volta accertati che l'implementazione della Backprop è corretta si può procedere ad addestrare la rete. Quello che si vuole ottenere è una rete che guardando ai dati accumulati negli anni, riesca a prevedere l'andamento del profitto del ristorante. Nel nostro dataset, ogni esempio è formato da una coppia \emph{<n. coperti, n. ore settimanali>} a cui è associata un'uscita \emph{desiderata}. La rete cercherà di modificare la sua struttura interna (i.e. i pesi sinaptici) "creando" una funzione - la rete stessa rappresenta questa funzione - per riprodurre in maniera più precisa possibile quest'associazione. L'apprendimento sarà quindi di tipo \emph{supervisionato}.
%% parlare del supervised learning e aggiungere grafico %%
\subsection{Apprendimento supervisionato}
\label{subsec:supervised}
Con questo termine s'intende l'allenamento di un sistema tramite una serie di esempi ideali; l'insieme di questi esempi è chiamato
\emph{training set}. Il sistema impara quindi ad approssimare una funzione non nota a priori a partire da una serie di coppie ingresso-uscita: per ogni input in ingresso gli si comunica l'output desiderato.
L'apprendimento consiste nella capacità del sistema – tramite la funzione generata con l'allenamento – di generalizzare a nuovi esempi: avendo in ingresso dati non noti deve poter predire in modo corretto l'output desiderato.
Matematicamente parlando, questi esempi non noti sono punti del dominio che non fanno parte dell'insieme degli esempi di training.
Esistono diversi algoritmi di apprendimento supervisionato ma tutti condividono una caratteristica: l'addestramento
viene eseguito mediante la minimizzazione di una funzione di costo (si veda la sezione \ref{sec:backprop}).
\begin{figure}[h!]
 \centering
 \includegraphics[width=0.8\textwidth]{\teoria/supervised-2.jpg}
 \caption{Apprendimento supervisionato: schema generale}
 \label{fig:supervised}
\end{figure}
Nella \emph{Learning Rule} in figura \ref{fig:supervised} sono compresi 2 elementi: 
\begin{enumerate}
\item Calcolo della discrepanza tra l'uscita della rete e l'uscita desiderata e backpropagation;
\item La strategia di aggiornamento dei parametri.
\end{enumerate}

Riguardo all'ultimo punto ci sono diverse possibilità. 
\subsection{Discesa del gradiente}
%% magari aggiungere un punto sul SGD%% 
La loss function è una funzione che ha un numero di variabili pari al numero dei pesi della rete. Data la complessità - soprattutto in reti profonde molto più complesse di quella trattata in questo progetto - la ricerca del minimo non è semplice; si corre il rischio di rimanere bloccati in plateau o minimi locali. Per questo, si applicano metodi a raffinamento iterativo: si parte da una soluzione iniziale e si cerca di migliorarla ad ogni ciclo. Questo metodo è conosciuto come \emph{Stochastig Gradient Descent} \parencite{WSGD}.
Calcolando il gradiente $\nabla J$ si conosce la direzione di massima variazione quindi ci si sposta - lungo l'opposto di questa direzione, l'antigradiente - di una quantità pari a $\eta$. Questo parametro si chiama \emph{learning rate} e regola appunto la velocità dell'apprendimento. 

Tornando alla strategia di aggiornamento dei parametri, la più intuitiva è chiamata \emph{"Vanilla"}: 
\begin{equation}
W_{t+1} = W_{t} - \eta \nabla J(W_t)
\end{equation}
Questa regola di apprendimento però soffre di alcuni problemi: 
\begin{itemize}
\item Effettua uno spostamento pari a $\eta$ sia per features frequenti che non. Questo problema è noto come \emph{"sparsità delle features"}.
\item $\eta$ è una costante e non è detto che garantisca la convergenza. Si potrebbe "saltare" da un lato all'altro del punto di minimo senza mai trovarlo. 
\item Come detto sopra, i punti di sella e plateau causano problemi. In questo caso il gradiente è nullo e quindi l'aggiornamento dei pesi si azzera, fermando l'apprendimento. 
\end{itemize}

Per ovviare a questo ed altri problemi, sono stati studiati numerosi metodi. La prossima sezione ne elenca qualcuno utilizzato per questo progetto. 

\section{Ottimizzazione: diverse tecniche}
L'ottimizzazione dell'apprendimento delle reti neurali è un argomento vasto ed impervio, ma molto importante. Con addestramenti che possono durare mesi a seconda del tipo di apprendimento, sono nate, in relativamente breve tempo, moltissimi metodi di ottimizzazione. Qui si faranno solo dei cenni ai metodi utlizzati in questo progetto. 

Prima di elencare tecniche più evolute dell'aggiornamento Vanilla, occorre precisare alcuni punti dello Stochastic Gradient Descent, essendo quest'ultimo il punto di partenza di ogni altro metodo. 

%SGD ASGD LBGFS ADAM
\begin{itemize}
\item \textsc{SGD}: lo \emph{Stochastic Gradient Descent} è, come lo descrive il nome stesso, una versione stocastica della discesa del gradiente. La discesa del gradiente standard non è scalabile, poiché il gradiente da calcolare tiene conto dell'errore quadratico calcolato \emph{su ogni singolo esempio del dataset}. Come detto poco fa, la loss function ha già di per sé un numero di variabili proporzionale alla complessità della rete; quando il dataset è molto largo, ed è tipico per problemi reali di machine learning, questo calcolo diventa inefficiente. L'SGD risolve questo problema approssimando l'operazione \emph{"gradiente $\rightarrow$ aggiornamento pesi"} da tutto il dataset al singolo esempio. Questo rende l'approssimazione molto inaccurata, ma molto veloce da elaborare. Nonostante le oscillazioni dovute all'inaccuratezza, questo metodo consente nella pratica, dopo molte iterazioni, di trovare il minimo globale. Questo è soprattutto vero per problemi su larga scala. 

Un compromesso tra il metodo standard e quello completamente stocastico è l'utilizzo di \emph{"mini-batch"}, cioè piccoli sottoinsiemi del dataset su cui viene eseguita l'iterazione di apprendimento. Se il dataset è abbastanza eterogeneo ed è ordinato in maniera randomica, il mini-batch approssima abbastanza bene l'intero dataset; di conseguenza, l'approssimazione sarà più verosimile al calcolo dell'intero gradiente, aumentando quindi l'accuratezza senza intaccare la velocità del calcolo. 
\item \textsc{Momentum \& NAG}: Il \emph{"Momentum"} controlla la quantità d'inerzia nella modifica dei pesi sinaptici, memorizzando nell'equazione la variazione $\Delta W$ precedente:
$$
W_{t} = W_{t-1} - \eta \nabla J(W_{t})- \underbrace{\mu\nabla 
J(W_{t-1})} 
$$ 
In questo modo si riducono le oscillazioni nella ricerca della soluzioni permettendo di usare learning rate più alti. È ispirato dal momento nella Fisica, considerando il vettore dei pesi come una particella che viaggia in uno spazio parametrico e acquisice velocità nella discesa. \\
Il \emph{Nesterov Accelerated Gradient} è una versione più raffinata del Momentum update, che elabora una correzione della traiettoria calcolando il gradiente \emph{dopo} aver fatto la somma con il gradiente accumulato precedentemente. Per aiutare a capire la differenza tra i due, si faccia riferimento alla figura \ref{fig:nag}.
\begin{figure}[h!]
 \centering
 \includegraphics[width=0.8\textwidth]{\teoria/nesterov.jpg}
 \caption{Confronto tra il momentum classico e NAG}
 \label{fig:nag}
\end{figure}
\item \textsc{BFGS}: l'algoritmo \emph{Broyden-Fletcher-Goldfarb-Shanno} fa parte della famiglia "Quasi-Newton"  \parencite{Wquasinewton}. Questo metodo supera i limiti della discesa del gradiente classica facendo una stima della matrice Hessiana, ovvero della curvatura della superficie della funzione di costo $J$. Utilizzando questa stima compie degli spostamenti più informati verso la discesa. Nella pratica si usa una versione efficiente che consuma meno memoria, detta Limited-BFGS \parencite{WLBFGS}. 

\item \textsc{Adam}: l'\emph{Adaptive Moment Estimation} cerca di ovviare ai problemi dell'aggiornamento vanilla modificando il learning rate in maniera diversa per ogni parametro e a seconda dello stadio dell'apprendimento. Fa parte quindi della famiglia dei metodi \emph{adattivi}. In particolare, l'algoritmo calcola la media con decadimento esponenziale del gradiente e del quadrato del gradiente; i parametri $\beta_1$ e $\beta_2$ controllano il decadimento di queste medie mobili. Gli autori forniscono i valori consigliati per questi parametri, che sono infatti i valori di default anche in ogni framework che supporta Adam \parencite{WAdam}.
\end{itemize}
\\
Per riuscire effettivamente ad addestrare la rete, si è utilizzato un package di ottimizzazione di Torch: \texttt{optim}. Quest'ultimo fornisce il supporto a tutte (e più) le tecniche di ottimizzazione viste prima. 
Si può quindi procedere all'addestramento della rete. Utilizzando un \emph{logger} per mantenere tutti i dati sul training, si possono successivamente plottare le diverse curve di apprendimento per confrontare i diversi metodi. \\
Definita quindi una classe \texttt{Trainer}, si definisce un metodo per addestrare la rete che astrae dalla tecnica di ottimizzazione utilizzata. 
\begin{lstlisting}[language={[5.2]Lua}]
Trainer = class('Trainer')
function Trainer:__init(NN)
  --Make Local reference to network:
  self.N = NN
end

--Let's train!
function Trainer:train(X, y)
   --variables to keep track of the training
   local neval = 0
   --get initial params
   params0 = self.N:getParams()
   -- create closure to evaluate f(X) and df/dX
   -- this is requested by the API of the optim package
   local feval = function(params0)
      local f = self.N:costFunction(X, y)
      print(f)
      local df_dx = self.N:computeGradients(X, y)
      neval = neval + 1
      logger:add{neval, f} --,timer:time().real}
      return f, df_dx
   end
   if optimMethod == optim.cg then
      newparams,_,_ = optimMethod(feval, params0, optimState)
   else
      for i=1,opt.maxIter do
         newparams,_,_ = optimMethod(feval, params0, optimState)
         self.N:setParams(newparams)
      end
   end
end
\end{lstlisting}
\\
Procedendo iterativamente con le diverse tecniche di ottimizzazione si ottiene il risultato in figura \ref{fig:comparison}. 

%INSERIRE GRAFICI
\begin{figure}[h!]
 \centering
 \includegraphics[width=1.0\textwidth]{\path/comparison.png}
 \caption{Confronto dei metodi di ottimizzazione durante il training}
 \label{fig:comparison}
\end{figure}
\\
Dal grafico si evidenzia come le premesse teoriche dei vari metodi siano state pienamente rispettate. Nonostante il problema sia molto semplice se comparato ai reali problemi di \emph{deep learning}, già su un dataset ed un numero di iterazioni così limitato si nota la superiorità di un metodo verso il precedente. In particolare, nel corso \texttt{"CS231"} di deep learning applicato alla visione artificiale di Stanford \parencite{WCS231adam}, sviluppato da Andrej Karpathy et. al, si raccomanda Adam come metodo di default per applicazioni di deep learning: 
\begin{quote}
\emph{In practice Adam is currently recommended as the default algorithm to use. However, it is often also worth trying SGD+Nesterov Momentum as an alternative}.
\end{quote}
\\
La rete è ora addestrata: dando in ingresso la matrice $X$ di input si avranno in output previsioni molto più accurate, come mostrato in figura \ref{fig:previsioni}. 
\begin{figure}[h!]
 \centering
 \includegraphics[width=0.5\textwidth]{\path/previsioni.jpg}
 \caption{L'output della rete $\hat{y}$ è vicino all'output desiderato $y$}
 \label{fig:previsioni}
\end{figure}

%--------------------------------------------------------------------%--------------------
%	SECTION 6
%--------------------
%--------------------------------------------------------------------
\section{Overfitting}
Uno dei problemi più comuni dell'apprendimento automatico è \emph{l'Overfitting}. Esso si verifica laddove il modello creato per fare predizioni risulta troppo complesso e troppo "legato" al solo training set, di cui apprende anche i rumori: la rete è quindi incapace di \emph{generalizzare} ad esempi ancora non visti e, nonostante l'accuratezza estremamente alta sul training set, darà delle pessime predizioni. Questo può essere dovuto a diversi fattori, di cui i più classici sono: 
\begin{itemize}
\item il modello ha troppi parametri rispetto al numero di osservazioni; 
\item il dataset è costituito da troppi pochi esempi; 
\item l'addestramento è stato fatto troppo a lungo.
\end{itemize}
\\
In figura \ref{fig:regularization} è mostrato un esempio di 2 modelli statistici con complessità molto diverse. In un primo istante la curva polinomiale avrà risultati decisamente migliori sul training set, salvo poi veder capovolgere la situazione sul test set.
\begin{figure}[h!]
 \centering
 \includegraphics[width=0.6\textwidth]{\path/reg-grafico.jpg}
 \caption{La funzione polinomiale ha un errore nullo sul dataset laddove la funzione lineare invece lo ha del 100\%. Tuttavia, la curva è eccessivamente complessa ed affetta da rumore;avrà quindi cattive capacità di generalizzazione. La retta, al contrario, approssima molto meglio i punti della distribuzione sottostante. Se definiamo questi punti come il test set, allora la retta avrà un'accuratezza maggiore.}
 \label{fig:regularization}
\end{figure}
\subsection{Rilevare l'overfitting}
Per rilevare se è avvenuto o meno overfitting durante l'apprendimento; si possono dare in ingresso al percettrone esempi ancora non visti e controllare "ad occhio" se le predizioni hanno senso. Per averne la certezza però, bisogna dividere il dataset in un training set ed un \emph{test set} su cui testare la rete durante l'apprendimento. Sugli esempi del test set non si farà backpropagation e non avverrà quindi apprendimento; si utilizzeranno solo per sapere quanto è corretto l'addestramento. \\
Testando la rete \emph{durante} l'apprendimento si può capire il preciso momento in cui avviene l'overfitting. Per farlo, occorre plottare sullo stesso grafico l'errore sul training set e sul test set. 

Dopo aver definito arbitrariamente alcuni esempi di testing, si modifica l'algoritmo di training: 
\begin{lstlisting}[language={[5.2]Lua}]
--Need to modify trainer class a bit to check testing error during training:
function Trainer:train(trainX, trainY, testX, testY)
   --variables to keep track of the training
   local neval = 0

   params0 = self.N:getParams()
   -- create closure to evaluate f(X) and df/dX
   local feval = function(params0)
      local f = self.N:costFunction(trainX, trainY)
      local test = self.N:costFunction(testX, testY)
      --printing training and testing error
      print(f..' '..test)
      local df_dx = self.N:computeGradients(trainX, trainY)
      neval = neval + 1
      --logging both training and testing data
      logger:add{neval, f} --,timer:time().real}
      testLogger:add{neval, test}

      return f, df_dx
   end

   if optimMethod == optim.cg then
      newparams,_,_ = optimMethod(feval, params0, optimState)
   else
      for i=1,opt.maxIter do
         newparams,_,_ = optimMethod(feval, params0, optimState)
         self.N:setParams(newparams)
      end
   end
end
\end{lstlisting}

Si addestra la rete e si loggano i valori di training e di testing: 
\begin{lstlisting}[language={[5.2]Lua}]
--now let's train and check where exactly the net is Overfitting
nn = Neural_Network(2,3,1)

init_params = nn:getParams()
logtrain = 'train.log'
logtest = 'test.log'
logger = optim.Logger(logtrain)
testLogger = optim.Logger(logtest)

trainer = Trainer(nn)
trainer:train(trainX, trainY, testX, testY)
\end{lstlisting}
\bigskip
\bigskip
Andando infine a plottare i dati così ottenuti, si osserva la presenza di overfitting in figura \ref{fig:overfitting}.
\newpage
\begin{figure}[h!]
 \centering
 \includegraphics[width=0.8\textwidth]{\path/overfitting.png}
 \caption{La rete mostra overfitting. Dopo l'iterazione 100 le performance sul test set iniziano ad essere sbagliate. Attorno alla 250 il modello diventa troppo complesso e le predizioni sono pessime}
 \label{fig:overfitting}
\end{figure}


\subsection{Contromisure}
Vi sono diverse contromisure contro l'overfitting, che seguono le cause più comuni: 
\begin{itemize}
\item \emph{Dataset più ampio}: una regola empirica consiste nell'avere circa 10 esempi per ogni parametro della rete. Nel caso del MLP di questo progetto i pesi sono 9, quindi secondo questa regola sono necessari 90 esempi. È ovvimente impossibile, poiché Loris ha collezionato solamente un limitato numero di esempi. 
\item \emph{K-fold cross-validation}: si divide il dataset in K set di eguale misura, ad ogni iterazione uno dei set viene utilizzato come test-set ed il resto per formare il training e validation set (si veda lez.1 del corso di Machine Learning per PhD di M. Lippi\prencite{WLippi}). Anche questa procedura non è applicabile in questo caso. 
\item \emph{Dropout}: il dropout è una tecnica di regolarizzazione maggiormente utilizzata nelle reti profonde, nella quale ad ogni iterazione si "spengono" alcuni neuroni (settando i pesi che li collegano a 0). Questo fa sì che neuroni vicini non si specializzino tutti a riconoscere le stesse caratteristiche, risultando in un minore overfitting. Una spiegazione approfondita esula dallo scopo di questo elaborato, per cui si suggerisce la consultazione del paper di Srivastava, Hinton et. al \parencite{Dropout}.  
\item \emph{Arresto anticipato}: una volta rilevato quando avviene precisamente l'overfitting, si arresta l'apprendimento \emph{prima} di quel momento. Osservando la figura \ref{fig:overfitting} si potrebbe pensare di fermare le iterazioni alla n.100 (o comunque prima della n. 250) e funzionerebbe. Così facendo, si ha una maniera piuttosto semplice di risolvere il problema. C'è uno svantaggio però: se la rete non è ancora ben addestrata non c'è possibilità di migliorarla ulteriormente, poiché ogni nuova iterazione causerebbe overfitting. Per questo, è stato usato l'ultimo metodo qui presentato. 
\item \emph{Regolarizzazione}: consiste nell'aggiungere un termine alla funzione di costo $J$ che penalizza modelli troppo complessi \parencite{WLippi}. Questo termine, $\lambda R(W)$ è regolato da un parametro $\lambda$ che definisce l'intensità della regolarizzazione. 
\end{itemize}\\
Un comune metodo di regolarizzazione è la \emph{L2 Regularization}: si implementa aggiungendo alla funzione di costo il quadrato dei tensori dei pesi della rete. In questa maniera si tengono i valori di tutti i pesi piuttosto bassi evitando che si specializzino troppo sui dati del training set. Di seguito il codice dei metodi della rete che devono essere modificati:  
\begin{lstlisting}[language={[5.2]Lua}]
--[[
## Introducing a Regularization term to mitigate overfitting ## 
Lambda will allow us to tune the relative cost: 
higher values of Lambda --> bigger penalties for high model complexity 
--]]

--so, the new Neural_Network class now becomes:
Neural_Network = class(function(net, inputs, hiddens, outputs, lambda)
      net.inputLayerSize = inputs
      net.hiddenLayerSize = hiddens
      net.outputLayerSize = outputs
      net.W1 = th.randn(net.inputLayerSize, net.hiddenLayerSize)
      net.W2 = th.randn(net.hiddenLayerSize, net.outputLayerSize)

      --regularization parameter
      net.lambda = lambda
   end)

function Neural_Network:costFunction(X, y)
   --Compute the cost for given X,y, use weights already stored in class
   self.yHat = self:forward(X)
   J = 0.5 * th.sum(th.pow((y-yHat),2))/X:size()[1] + 
            (self.lambda/2) * (th.sum(th.pow(self.W1,2)) + th.sum(th.pow(self.W2, 2)))

   return J
end

function Neural_Network:d_CostFunction(X, y)
   --Compute derivative wrt to W and W2 for a given X and y
   self.yHat = self:forward(X)
   delta3 = th.cmul(-(y-self.yHat), self:d_Sigmoid(self.z3))
   --Add gradient of regularization term:
   dJdW2 = th.mm(self.a2:t(), delta3)/X:size()[1] + self.lambda*self.W2

   delta2 = th.mm(delta3, self.W2:t()):cmul(self:d_Sigmoid(self.z2))
   --Add gradient of regularization term:
   dJdW1 = th.mm(X:t(), delta2)/X:size()[1] + self.lambda*self.W1

   return dJdW1, dJdW2
end
\end{lstlisting}
\linebreak
Eseguendo ora l'addestramento si ottengono risultati in figura \ref{fig:reg-plot}. Si può osservare come la regolarizzazione sia stata efficace e le 2 curve siano pressoché identiche. Si noti anche che, data la modesta complessità del problema, dopo circa la 130esima iterazione si è già raggiunto un minimo ed il processo di apprendimento non migliora ulteriormente. 
\begin{figure}[h!]
 \centering
 \includegraphics[width=0.65\textwidth]{\path/regularization.png}
 \caption{Dopo aver applicato la L2 regularization, la rete non è più affetta da overfitting.}
 \label{fig:reg-plot}
\end{figure}

%%inserire grafico ultimo

%--------------------------------------------------------------------%--------------------
%	SECTION 7
%--------------------
%--------------------------------------------------------------------
\newpage
\section{Risultati}
In questo capitolo si è visto come costruire da zero l'architettura di un percettrone multi-strato con paradigma OOP; come implementare la backpropagation e controllare numericamente che funzioni in maniera corretta. Successivamente si sono visti i principali algoritmi per ottimizzare l'apprendimento supervisionato di una rete neurale. Si è poi affrontato il problema dell'overfitting: come rilevarlo in maniera precisa ed un introduzione alle tecniche più comuni per risolverlo. \\
Infine, si è ottenuto una rete neurale capace di predire in maniera corretta il profitto di un ristorante in base al numero di coperti ed il numero di ore di apertura settimanali.  
% Chapter Template

\chapter{Reti Neurali Convoluzionali} % Main chapter title
\label{Capitolo3}
\def \teoria {Figures/teoria}
\def \path	 {Figures/C3}
%--------------------------------------------------------------------%	SECTION 1
%--------------------------------------------------------------------
%% da completare
In questo capitolo si introduce una panoramica generale sulle reti neurali convoluzionali. Essendo un argomento vasto, una trattazione teorica approfondita sarebbe materia di una tesi di laurea, ragion per cui gli argomenti sono introdotti con lo scopo di avere un'infarinatura per comprendere le applicazioni sviluppate nei capitoli successivi. 
\section{Breve introduzione}
La reti neurali convoluzionali, alle quali ci riferiremo con l'abbrevazione \emph{CNN} - dall'inglese \emph{Convolutional Neural Network}, sono un'evoluzione delle normali reti artificiali profonde caratterizzate da una particolare architettura estremamente vantaggiosa per compiti visivi (e non), che le ha rese negli anni molto efficaci e popolari. Sono state ispirate dalle ricerche biologiche di Hubel e Wiesel i quali, studiando il cervello dei gatti, avevano scoperto che la loro corteccia visiva conteneva una complessa struttura di cellule. Quest'ultime erano sensibili a piccole parti locali del campo visivo, detti campi recettivi \emph{(receptive fields)}. Agivano quindi da filtri locali perfetti per comprendere la correlazione locale degli oggetti in un'immagine. Essendo questi sistemi i più efficienti in natura per la comprensione delle immagini, i ricercatori hanno tentato di simularli. 

%--------------------------------------------------------------------
%	SECTION 2
%--------------------------------------------------------------------

\section{Architettura}
Le CNN sono reti neurali profonde costituite da diversi strati che fungono da estrattori delle features ed una rete completamente connessa alla fine, che funge da classificatore, come raffigurato in figura \ref{fig:cnn1}. \\
\begin{figure}[h!]
 \centering
 \includegraphics[width=1.0\textwidth]{\path/CNN-expl.png} 
 \caption{Architettura di una CNN che classifica segnali stradali: si evidenzia la divisione tra gli strati che fungono da feature extractor ed il classificatore finale}
 \label{fig:cnn1}
\end{figure}
Questi strati in cui si estraggono le caratteristiche delle immagini sono detti strati di convoluzione, e sono generalmente seguiti da una funzione non lineare e un passo di \emph{pooling}. Vi possono poi essere degli strati di elaborazione dell'immagine, come quello di normalizzazione del contrasto, si veda \fig{ref:cnn2}.
%%% ricollocare la figura %%% 
\begin{figure}[h!]
 \centering
 \includegraphics[width=1.0\textwidth]{\path/CNN-features.png} 
 \caption{I diversi strati tipici di una CNN}
 \label{fig:cnn2}
\end{figure}
Convoluzione e pooling hanno come scopo quello di estrarre le caratteristiche, mentre l’unità non lineare serve a rafforzare le caratteristiche più forti e indebolire quelle meno importanti, ovvero quelle che hanno stimolato meno i neuroni (si dice che fa da “squashing”). \\
Sempre dalla figura \ref{fig:cnn1}, possiamo inoltre notare che, per ogni immagine in input, corrispondono nei vari strati, diversi gruppi di immagini, che vengono chiamate \emph{feature maps}. Le feature maps sono il risultato dell'operazione di convoluzione svolta tramite un banco di filtri, chiamati anche kernel, che altro non sono che delle matrici con dei valori utili a ricercare determinate caratteristiche nelle immagini.\\
Infine, terminati i convolutional layers, le feature maps vengono “srotolate” in vettori e affidate ad una rete neurale "classica" che esegue la classificazione finale. 

Il numero di strati di convoluzione è arbitrario. Inizialmente, quando le CNN divennero famose grazie a Y.LeCun, che addestrò una CNN chiamata \emph{"LeNet5"} al riconoscimento dei numeri \parencite{lenet}, questo numero era compreso tra 2-5. Nel 2012, Alex Krizhevsky et al \parencite{imagenet2012} addestrarono una rete costituita da 5 strati di convoluzione, 60 milioni di parametri e 650 mila neuroni. Ottennero la migliore percentuale d'errore al mondo sul dataset ImageNet ILSVRC-2010, contenente 1,2 milioni di immagini divise in 1000 categorie. \\
Da allora le cose si sono evolute con una velocità disaramente, e l'ImageNet challenge del 2015, è stata vinta da una rete con 152 strati \parencite{resnet}. Nel capitolo \ref{Capitolo5} si farà un confronto tra quest'ultima rete, soprannominata \emph{"ResNet"} e la capostipite LeNet5 su un task di classificazione.   
\subsection{Strato di Convoluzione}
Per comprendere appieno quello che avviene in una CNN, occorre introdurre il concetto di convoluzione fra matrici, e capire come questo sia importante per applicare dei filtri ad un'immagine digitale.

Un’immagine digitale può essere considerata come una matrice A di dimensione $M×N$ valori reali o discreti. Ogni valore della matrice prende il nome di pixel e i suoi indici sono anche chiamati coordinate: ogni pixel $A(m,n)$ rappresenta l’intensità nella posizione indicata dagli indici. \\ 
Si definisce “filtro” o “kernel” una trasformazione applicata ad un’immagine. Come detto prima, questi filtri sono a loro volta della matrici; la trasformazione quindi si effettua appunto tramite un'operazione di convoluzione tra l'immagine in ingresso ed il filtro.
La convoluzione, discreta nel caso di immagini digitali, si può definire come: 
%% INSERIRE EQUZIONEI %%% 

Ogni pixel di $y[m,n]$ è così il risultato di una somma pesata tramite $h[m,n]$ della sottoregione che ha centro nel pixel indicato dalle coordinate m,n. Un esempio di convoluzione è rappresentato in figura \ref{fig:convolution}. 
\begin{figure}[h!]
 \centering
 \includegraphics[width=1.0\textwidth]{\path/convolution.png} 
 \caption{Convoluzione con un kernel: primi due step}
 \label{fig:convolution}
\end{figure}
Nei convolutional layers viene quindi fatta un'operazione di convoluzione tra l'immagine/i in ingresso
e un numero arbitrario K di filtri. Questi filtri hanno valori tali da ottenere in uscita – tramite convoluzione – un riconoscimento di determinate caratteristiche. \\
I valori dei filtri sono all'inizio scelti casualmente, e vengono poi migliorati ad ogni iterazione mediante l'algoritmo di backpropagation, visto nel Capitolo \ref{Capitolo2}. Così facendo la rete addestra i suoi filtri ad estrarre le features più importanti degli esempi del training set; cambiando training set i valori dei
filtri saranno diversi.
Ad esempio, i valori dei filtri di una rete allenata con immagini di pali verticali saranno diversi da quella allenata con immagini di palloni da calcio; nel primo caso i valori saranno calibrati per riconoscere lunghi orientamenti verticali, mentre
nel secondo per riconoscere i confini sferici. 

Nelle reti convoluzionali quindi, l'algoritmo di Backpropagation migliora i valori dei filtri della rete, è lì quindi che si accumula l'apprendimento. I neuroni, in queste reti, devono intendersi come i singoli filtri.\\
\\
Vi sono diversi \emph{hyperparameters} da settare manualmente negli strati di convoluzione: 
\begin{enumerate}
\item la misura del filtro $F$: chiamato anche \emph{receptive field}. Ogni filtro cerca una determinata caratteristica in un'area locale dell'immagine, la sua misura quindi è il campo recettivo del singolo neurone. Tipicamente sono 3x3, 5x5 o 7x7.

\item il numero $K$ di filtri: per ogni strato, questo valore definisce la profondità dell'output dell'operazione di convoluzione. Infatti, mettendo una sopra l'altra le feature maps, si ottiene un cubo in cui ogni "fetta" è il risultato dell'operazione tra l'immagine in ingresso ed il corrispettivo filtro. La profondità di questo cubo dipende appunto dal numero dei filtri. 

\item lo \emph{"Stride"} S: definisce di quanti pixel si muove il filtro della convoluzione ad ogni passo. Se lo stride è settato a 2, il filtro salterà 2 pixel alla volta, producendo quindi un output più piccolo. 

\item il \emph{"padding"} P: definisce la misura con la quale si vuole aggiungere degli "0" all'input per preservare la dimensione in output. In generale, quando lo stride S=1, un valore di  $P = (F - 1)/2$ garantisce che l'output avrà le stesse dimensioni dell'input. 
\end{enumerate}

%% pezzo sulla dimensione in output %% 
Quando si elaborano delle immagini con le CNN si hanno generalmente in ingresso degli input tridimensionali, caratterizzati dall'altezza H_1, l'ampiezza W_1 e il numero di canali di colore D_1. Conoscendo i parametri sopra specificati si può calcolare la dimensione dell'output di un layer di convoluzione: 
\begin{align*}
H_2 = (H_1 - F + 2P)/S + 1
W_2 = (H_1 - F + 2P)/S + 1
D_2 = K
\end{align*}
%% astrazione %% 

Gli strati di convoluzioni mostrano molte proprietà interessanti, di cui qui ne cito 2. \\
In primo luogo, se l'immagine in input viene traslata, l'output della feature map sarà traslato della stessa quantità ma rimarrà invariato altrove. Questa proprietà è alla base della robustezza rispetto alle traslazioni e alle distorsioni dell'immagine in ingresso; in secondo luogo, mettendo in fila diversi strati di convoluzioni si ottiene una rete capace di avere una comprensione più "astratta" dell'immagine in ingresso. Il primo strato di convoluzione si occupa di estrarre features direttamente dai pixel grezzi dell'immagine e li memorizza nelle feature maps. Questo output diviene poi l'input di un successivo livello di convoluzione, il quale andrà a fare una seconda estrazione delle caratteristiche, combinando le informazioni dello strato precedente. Da questa astrazione a più livelli deriva una maggior comprensione delle features. 


\subsection{Strato di ReLU}
Nel capitolo \ref{Capitolo2} si è detto che la funzione sigmoide non era la più efficiente. Difatti, negli anni si è stabilita con sicurezza la \emph{Rectified Linear Unit} (ReLU) come funzione d'attivazione più efficace. La ReLU è più verosimile alla modalità di attivazione biologica dei nostri neuroni\parencite{Relu}, ed è definita come: 
$$
f(x) = max(0,x)
$$ 
Y. LeCun ha dichiarato che la ReLU è inaspettatamente \emph{“l'elemento
singolo più importante di tutta l'architettura per un sistema di riconoscimento”}. Questo può essere dovuto principalmente a 2 motivi:
\begin{enumerate}
\item la polarità delle caratteristiche è molto spesso irrilevante per riconoscere gli oggetti;
\item la ReLU evita che quando si esegue pooling (sezione \ref{subsec:pooling}) due caratteristiche entrambe importanti ma con polarità opposte si cancellino fra loro
\end{enumerate}
\subsection{Strato di Pooling}
\label{subsec:pooling}

\subsection{Strato completamente connesso (FC)}
%% menzionare che adesso si usa la convoluzione %% 

%--------------------------------------------------------------------
%	SECTION 3
%--------------------------------------------------------------------

\section{Applicazioni e risultati}



% Chapter Template
\chapter{4 Tensor Decomposition} % Main chapter title
\label{Chapter4}
\def \path	{Figures/C4}
In questo capitolo verranno messi in pratica i concetti introdotti nel Capitolo \ref{Capitolo3}. In particolare, verrà addestrata una CNN su 2 dataset, MNIST e CIFAR. I risultati su quest'ultimo serviranno da confronto con ResNet, la rete utilizzata nel Capitolo 5. 
%--------------------------------------------------------------------
%	SECTION 1
%--------------------------------------------------------------------
\section{Background}}
%% intro 

%% SVD Applied on Fully Connected layer 

$$ (U_{n \times t}\Sigma_{t \times t}V^T_{m \times t})x + b $$ = $$ U_{n \times t} ( \Sigma_{t \times t}V^T_{m \times t} x ) + b $$


%--------------------------------------------------------------------
%	SECTION 2
%--------------------------------------------------------------------
\section{}


%--------------------------------------------------------------------
%	SECTION 3
%--------------------------------------------------------------------
\section{MNIST: preprocessing}


%%inserire codice per dataset MNIST %% 
%--------------------------------------------------------------------
%	SECTION 4
%--------------------------------------------------------------------
\section{MNIST: addestramento}


%--------------------------------------------------------------------
%	SECTION 5
%--------------------------------------------------------------------
\section{CIFAR: preprocessing}




%--------------------------------------------------------------------
%	SECTION 6
%--------------------------------------------------------------------
\section{CIFAR: addestramento}




%%%%%%%%%%%%%%%%%%%%%%%%%%%%%%%%%%%%%%%%%%%%%%%%%%
%%%%%%%%%%%%%%%%%%%%%%%%%%%%%%%%%%%%%%%%%%%%%%%%%%

%% Approximation with CPD 

\begin{equation}
$$ \sum_{r=1}^R K^x_r(i)K^y_r(j)K^s_r(s)K^t_r(t) $$.	
\end{equation}


%%%% forward pass con CPD %%% 
\begin{center}
\begin{align*}
	V(x, y, t) = \sum_i \sum_j \sum_s K(x-i, y-j, s, t)X(i, j, s) \\
		= \sum_r \sum_i \sum_j \sum_s K^x_r(x-i)K^y_r(y-i)K^s_r(s)K^t_r(t)X(i, j, s)\\
			= \sum_r K^t_r(t) \sum_i \sum_j K^x_r(x-i)K^y_r(y-i) \sum_s K^s_r(s) X(i, j, s) \tag{4}
\end{align*}
\end{center}


%% forward pass con Tucker 
The Tucker decomposition, also known as the Higher Order SVD  \emph{HOSVD} is a generalization of SVD for tensors: 

$$ K(i, j, s, t) = \sum_{r_1=1}^{R_1}\sum_{r_2=1}^{R_2}\sum_{r_3=1}^{R_3}\sum_{r_4=1}^{R_4}G_{r_1, r_2, r_3, r_4} * K^x_{r1}(i)K^y_{r2}(j)K^s_{r3}(s)K^t_{r4}(t) $$

The Tucker decomposition has a property that's useful for our purposes: it doesn't have to be applied along all modes (axis) of the tensors. We have seen how CPD decomposes the kernel also spatial-wise; this acts
pretty aggressively on the number of parameters and since the convolutional kernels are small in most of the recent implementations ($3x3$ or $5x5$) it arguably doesn't save a lot of computation. Hence, we can skip
this decomposition with Tucker, going for what is known as a Tucker-2 decomposition: 

\begin{equation}
\label{eq:tucker-def}
	K(i, j, s, t) = \sum_{r_3=1}^{R_3}\sum_{r_4=1}^{R_4} \textbf{G} {i,j,r_3, r_4}(j)K^s_{r3}(s)K^t_{r4}(t) 
\end{equation} 

Using equation \ref{eq:tucker-def} and plugging into the formula for the convolutional forward pass, we obtain the new equation for the Tucker convolutional forward pass: 

$$ V(x, y, t) = \sum_i \sum_j \sum_sK(x-i, y-j, s, t)X(i, j, s) $$

$$ V(x, y, t) = \sum_i \sum_j \sum_s\sum_{r_3=1}^{R_3}\sum_{r_4=1}^{R_4}\sigma_{(x-i)(y-j) r_3 r_4}K^s_{r3}(s)K^t_{r4}(t)X(i, j, s) $$

$$ V(x, y, t) = \sum_i \sum_j \sum_{r_4=1}^{R_4}\sum_{r_3=1}^{R_3}K^t_{r4}(t)\sigma_{(x-i)(y-j) r_3 r_4} \sum_s\ K^s_{r3}(s)X(i, j, s) $$ 



\begin{lstlisting}[language={[5.2]Lua}]
----- preprocess/normalize train/test sets -----
print '<trainer> preprocessing data (color space + normalization)'

-- preprocess trainSet
normalization = nn.SpatialContrastiveNormalization(1, image.gaussian1D(7))
for i = 1,trainData:size() do
   --rgb -> yuv
   local rgb = trainData.data[i]
   local yuv = image.rgb2yuv(rgb)
   
   -- normalize Y locally:
   yuv[1] = normalization(yuv[{{1}}])
   trainData.data[i] = yuv
end
-- normalize U globally:
mean_u = trainData.data[{ {},2,{},{} }]:mean()
std_u = trainData.data[{ {},2,{},{} }]:std()
trainData.data[{ {},2,{},{} }]:add(-mean_u)
trainData.data[{ {},2,{},{} }]:div(-std_u)

-- normalize V globally:
mean_v = trainData.data[{ {},3,{},{} }]:mean()
std_v = trainData.data[{ {},3,{},{} }]:std()
trainData.data[{ {},3,{},{} }]:add(-mean_v)
trainData.data[{ {},3,{},{} }]:div(-std_v)

--same applies to test set...
\end{lstlisting}



\begin{figure}[h!]
 \centering
 \includegraphics[width=1.0\textwidth]{\path/k-mode.png} 
  \caption{Example of k-mode multiplication on 3-dimensional tensor.}
 \label{fig:k-mode}
\end{figure}

\bigskip

\begin{figure}
\centering
\begin{subfigure}{.5\textwidth}
  \centering
 \includegraphics[width=1\textwidth]{\path/cifar-tanh.png} 
  \caption{Acc.= \textasciitilde 65\% f. di attivazione = TanH}
 \label{fig:training}
\end{subfigure}%
\begin{subfigure}{.5\textwidth}
  \centering
 \includegraphics[width=1\textwidth]{\path/cifar-relu.png} 
  \caption{Acc.= \textasciitilde 73\% f. di attivazione = ReLU}
 \label{fig:validation}
\end{subfigure}
\caption{Percentuali di accuracy a seconda della funzione d'attivazione. La ReLU produce indubbiamente risultati migliori.}
\label{fig:relu}
\end{figure}
\\
%%%%% NEW PAGE FOR THE NEW FIGURES %%%% 
\newpage
\pagebreak
\medskip
\newpage

\begin{figure}[h!]
 \centering
 \includegraphics[width=1.0\textwidth]{\path/tensor-mode.jpg} 
 \caption{Fibers and slices of a tensor: fibers is an equivalent term for a tensor mode.}
 \label{fig:tensor-fibers}
\end{figure}

\begin{figure}[h!]
 \centering
 \includegraphics[width=1.0\textwidth]{\path/tucker-pass.jpg} 
 \caption{Tucker-2  decompositions  for  speeding-up  a generalized convolution. Each box corresponds to a 3-way tensor $X, Z, Z^' and Y$ in equation (\ref{eq:tucker1}-\ref{eq:tucker3}). Arrows represent linear mappings 
and illustrate each scalar value on the right is computed. Red tube, green cube and blue tube correspond to 
1x1, dxd and 1x1 convolution respectively.}
 \label{fig:tucker-pass}
\end{figure}


\begin{figure}[h!]
 \centering
 \includegraphics[width=1.0\textwidth]{\path/CPD.jpg} 
 \caption{Tensor decompositions for speeding up a generalized convolution. Each box correspond to a feature map stack within a CNN, (frontal sides are spatial dimensions). Arrows show linear mappings and demonstrate how scalar values on the right are computed. Initial full convolution (A) computes each element of the target tensor as a linear combination of the elements of a 3D subtensor that spans a spatial d × d window over all input maps. 
Jaderberg et al. (B) approximate the initial convolution as a composition of two linear mappings in which the intermediate mpa stack has R  maps, being R the rank of the decomposition. Each of the two-components 
computes each target value with a convolution based on a spatial window of size dx1 or 1xd in all input maps. Finally, CP-decomposition (C) by Lebedev et al. approximates the convolution as a composition of four smaller convolutions: the first and the last components compute a standard 1x1 convolution that spans all input maps while the middle ones compute a 1D grouped convolution \textbf{only on one} input map.}
 \label{fig:cpd-pass}
\end{figure}

Tensor decompositions for speeding up a generalized convolution. Gray boxes corre-
spond to 3D tensors (map stacks) within a CNN, with the two frontal sides corresponding to spatial
dimensions. Arrows show linear mappings and demonstrate how scalar values on the right (small
boxes corresponding to single elements of the target tensor) are computed. Initial full convolu-
tion (a) computes each element of the target tensor as a linear combination of the elements of a
3D subtensor that spans a spatial d × d window over all input maps. Jaderberg et al. (2014a) (b)
approximate the initial convolution as a composition of two linear mappings with the intermediate
map stack having R maps (where R is the rank of the decomposition). Each of the two mappings
computes each target value based on a spatial window of size 1×d or d×1 in all input maps. Finally,
CP-decomposition (c) used in our approach approximates the convolution as a composition of four
convolutions with small kernels, so that a target value is computed based on a 1D-array that spans
either one pixel in all input maps, or a 1D spatial window in one input map.
\\

 
% Chapter Template
\chapter{Experimental Results \& Analysis} % Main chapter title
\label{Chapter5}
\def \teoria {Figures/teoria}
\def \path	 {Figures/C5}
\def \plots  {Figures/plots}

In chapter \ref{Chapter4} the theoretical background of tensor decomposition has been introduced. The way Tucker and CP decomposition can be applied on convolutional layers has suggested a specific micro-architecture, hence the TD-block has been introduced. Following this line, this chapter will show how to apply the TD-block as a building block and as a compression block for five different models. 


%--------------------------------------------------------------------
%	SECTION 1
%--------------------------------------------------------------------
\section{Experimental setup}
In this section, we break down the dataset and the five different architectures on which tensor decomposition has been applied. The core strategies of the experiments will be described and motivated. 


\subsection{Tools}
All the experiments have been executed in Ubuntu 17.10 \texttt{Intel(R) Core(TM) i7-4510U @2.00 GHz} cpu for testing and on a \texttt{NVIDIA GeForce 840M} with 2GB of memory for training. Most of the experiments have been implemented in \emph{PyTorch} \parencite{pytorch} motivated by its flexibility and fast prototyping. For example, network surgery and decomposition are more intuitive in PyTorch than other frameworks. Regarding experiment 5 instead, the implementation was done in TensorFlow\parencite{tensorflow} and Keras\parencite{keras}.  

About tensor decomposition, two frameworks have been adopted for the experiments: 
\begin{enumerate}
    \item \texttt{TensorLab} toolbox for MATLAB \parencite{WT	nsorlab}: very comprehensive tool for general tensor computations. It provides a plethora of algorithms and optimizations to compute tensor factorizations. \\ 
    There are also interesting tools for tensor visualization that could be useful for further investigations on the convolutional layer tensor structure. 
    
    \item \texttt{Tensorly} toolbox for Python \parencite{Wtensorly}: it is a light wrapping around NumPy but provides very few optimizations compared to TensorLab; for instance only ALS and HOOI to compute CP and Tucker respectively. Nevertheless, it gets the job done and easily embeddable into deep learning frameworks, as it is built on NumPy. \\
    However, for large convolutional layers with many activations (as the first ones) with large dense tensors it can stall or occur in memory errors (it saturates the RAM). 
\end{enumerate}

\subsubsection{Pipeline implementation}
The decomposition pipeline has been implemented as a python extensible class. This 'decomposer' class provides methods for CP, Tucker and Xavier decomposition with different configurations. It also features the possibility to manually select the desired compression ratio for a particular layer. The work supports Pytorch and Keras for now, but will be extended to TensorFlow in the near future.  

\subsection{Datasets}
Experiments have been conducted on two datasets. As for other compression techniques in literature, a classification dataset, CIFAR10, has been selected as the testbed for most experiments. \\
Furthermore, the TD-block design has been tested on the KITTI dataset, to learn automaticatic estimation of disparity maps. 


\subsubsection{CIFAR-10}
Although the CIFAR10 is not among the most challenging datasets, it provides a relatively fast way to test new ideas while also being not as trivial as MNIST. 


The dataset consists of 60000 color images of size 32×32 pixels divided into 10 different classes, with 6000 images per class. 50000 of the images will be used for the training process and 10000 will be used to test the network. \\
The 10 classes in the dataset are: airplane, automobile, bird, cat, deer, dog, frog,horse, ship and truck. All those classes are completely mutually exclusive, not as in other datasets. Some examples of those classes can be seen in figure \ref{fig:cifar}. 

The batch size for every training has been set to 32, as a common chioce for this dataset.

\subsubsection{KITTI}
The KITTI stereo/flow benchmark consists of 194 training image pairs and 195 test image pairs, saved in loss less png format. This dataset is provideds a set of real image pairs to evaluate stereo matching algorithms. Specifically, it will be used to compress CCNN \parencite{WCCNN}, a deep convnet that has been able to infer from scratch an effective confidence measure by only using as input cue the disparity map. 
\\

Since this model is trained on 9$\times $ patches of large input images, training will be performed only 20 images while 174 will act as a test set. Although the size of the patches is relatively small compared to the disparity map, it provides to CCNN enough cues to infer the degree of uncertainty for each point.

\subsubsection{LeNet1}
This model is a variation on the classic LeNet\parencite{lenet} model from Y.LeCun. It is composed by four convolution layers followed by ReLUs, and two fully-connected layers for the final classification. Max pooling and dropout completes the architecture, which is reported in details in table  \ref{tab:lenet1}. The total number of parameters is $1250858$, i.e. $\approx 1.25$ million. 
\newline 

This model will be used in order to test the TD-block micro-architecture effectiveness. 


\begin{comment}
% Please add the following required packages to your document preamble:
% \usepackage[table,xcdraw]{xcolor}
% If you use beamer only pass "xcolor=table" option, i.e. \documentclass[xcolor=table]{beamer}
\begin{table}[]
\centering
\caption{Architecture of LeNet-1}
\label{lenet1}
\begin{tabular}{|c|c|}
\hline
\textbf{LAYER} & \textbf{CHARACTERISTICS}                    \\ \hline
\rowcolor[HTML]{CBCEFB} 
CONV           & 32 filters 3$\times$3, padding=1, stride=1  \\ \hline
\rowcolor[HTML]{EFEFEF} 
ReLU           & -                                           \\ \hline
\rowcolor[HTML]{CBCEFB} 
CONV           & 32 filters 3 $\times$ 3, padding=0 stride=1 \\ \hline
\rowcolor[HTML]{EFEFEF} 
ReLU           & -                                           \\ \hline
\rowcolor[HTML]{FFCCC9} 
POOL           & pool size {[}2, 2{]} stride=2               \\ \hline
Dropout        & p=0.25                                      \\ \hline
\rowcolor[HTML]{CBCEFB} 
CONV           & 64 filters 3 $\times$ 3, padding=1 stride=1 \\ \hline
\rowcolor[HTML]{EFEFEF} 
ReLU           & -                                           \\ \hline
\rowcolor[HTML]{CBCEFB} 
CONV           & 64 filters 3 $\times$ 3, padding=0 stride=1 \\ \hline
\rowcolor[HTML]{EFEFEF} 
ReLU           & -                                           \\ \hline
\rowcolor[HTML]{FFCCC9} 
POOL           & pool size {[}2, 2{]} stride=2               \\ \hline
\rowcolor[HTML]{FFFFFF} 
Dropout        & p=0.25                                      \\ \hline
\rowcolor[HTML]{FBF1A2} 
FC             & 512 units                                   \\ \hline
\rowcolor[HTML]{FBF1A2} 
FC             & \#Classes (=10) units                       \\ \hline
\end{tabular}
\caption{Architecture of LeNet-1}
\label{tab:lenet1}
\end{table}
\end{comment}

Different models:  LeNet-1 (classic) LeNet-Conv (with fc layers converted in conv)  LeNet-2 (the one used in Zhang et al., metodo maragno) NIN (Network in Network used in Zhang et al.)  
CCNN (Poggi)


\subsubsection{LeNet-Conv}
The \texttt{LeNet-Conv} model is similar to the previous one with the difference of having convolutional layers as classifiers instead of fully-connected ones. The conversion has been applied as described in chapter \ref{Chapter3}. Additionally it features BatchNorm layers. The whole architecture is depicted in table \ref{tab:lenet-conv}. 

Therefore it has identical accuracy while being fully decomposable. The number of trainable parameters is also the same.  

% Please add the following required packages to your document preamble:
% \usepackage[table,xcdraw]{xcolor}
% If you use beamer only pass "xcolor=table" option, i.e. \documentclass[xcolor=table]{beamer}
\begin{table}[]
\centering
\begin{tabular}{|c|c|}
\hline
\textbf{LAYER} & \cellcolor[HTML]{FFFFFF}\textbf{CHARACTERISTICS}          \\ \hline
\rowcolor[HTML]{CBCEFB} 
CONV           & 32 filters 3$\times$3, padding=1, stride=1                \\ \hline
\rowcolor[HTML]{EFEFEF} 
ReLU           & -                                                         \\ \hline
\rowcolor[HTML]{C3EDF8} 
BNORM          & eps = 1e-5 momentum=0.1                                   \\ \hline
\rowcolor[HTML]{CBCEFB} 
CONV           & 32 filters 3 $\times$ 3, padding=0 stride=1               \\ \hline
\rowcolor[HTML]{EFEFEF} 
ReLU           & -                                                         \\ \hline
\rowcolor[HTML]{C3EDF8} 
BNORM          & eps = 1e-5 momentum=0.1                                   \\ \hline
\rowcolor[HTML]{FFCCC9} 
POOL           & pool size {[}2, 2{]} stride=2                             \\ \hline
\rowcolor[HTML]{CBCEFB} 
CONV           & 64 filters 3 $\times$ 3, padding=1 stride=1               \\ \hline
\rowcolor[HTML]{EFEFEF} 
ReLU           & -                                                         \\ \hline
\rowcolor[HTML]{C3EDF8} 
BNORM          & eps = 1e-5 momentum=0.1                                   \\ \hline
\rowcolor[HTML]{CBCEFB} 
CONV           & 64 filters 3 $\times$ 3, padding=0 stride=1               \\ \hline
\rowcolor[HTML]{EFEFEF} 
ReLU           & -                                                         \\ \hline
\rowcolor[HTML]{C3EDF8} 
BNORM          & eps = 1e-5 momentum=0.1                                   \\ \hline
\rowcolor[HTML]{FFCCC9} 
POOL           & pool size {[}2, 2{]} stride=2                             \\ \hline
\rowcolor[HTML]{C3EDF8} 
BNORM          & eps = 1e-5 momentum=0.1                                   \\ \hline
\rowcolor[HTML]{CBCEFB} 
CONV           & 512 filters $6 \times 6$, padding=0, stride=1             \\ \hline
ReLU           & -                                                         \\ \hline
\rowcolor[HTML]{C3EDF8} 
BNORM          & eps = 1e-5 momentum=0.1                                   \\ \hline
\rowcolor[HTML]{CBCEFB} 
CONV           & \#Classes (=10) filters $1 \times 1$, padding=0, stride=1 \\ \hline
\rowcolor[HTML]{C3EDF8} 
BNORM          & eps = 1e-5 momentum=0.1                                   \\ \hline
\end{tabular}
\caption{LeNet-Conv architecture. }
\label{tab:lenet-conv}
\end{table}



\subsection{LeNet-2}
This is another variation on LeNet, with fewer layers with bigger filter sizes and more channels, i.e. it is less deep but wider. This architecture is useful as it was also employed by Zhang et al. \parencite{zhang2015SVD} and hence can be a good measure of comparison between compression methods. 

Table \ref{tab:lenet2} summarizes the full architecture. 

% Please add the following required packages to your document preamble:
% \usepackage[table,xcdraw]{xcolor}
% If you use beamer only pass "xcolor=table" option, i.e. \documentclass[xcolor=table]{beamer}
\begin{table}[]
\centering
\begin{tabular}{|c|c|}
\hline
\textbf{LAYER} & \cellcolor[HTML]{FFFFFF}\textbf{CHARACTERISTICS} \\ \hline
\rowcolor[HTML]{CBCEFB} 
CONV           & 192 filters 5$\times$5 padding=2, stride=1       \\ \hline
\rowcolor[HTML]{EFEFEF} 
ReLU           & -                                                \\ \hline
\rowcolor[HTML]{FFCCC9} 
POOL           & pool size {[}2, 2{]} stride=2                    \\ \hline
\rowcolor[HTML]{CBCEFB} 
CONV           & 128 filters 5 $\times$ 5, padding=2 stride=1     \\ \hline
\rowcolor[HTML]{EFEFEF} 
ReLU           & -                                                \\ \hline
\rowcolor[HTML]{FFCCC9} 
POOL           & pool size {[}2, 2{]} stride=2                    \\ \hline
\rowcolor[HTML]{CBCEFB} 
CONV           & 256 filters 3 $\times$ 3, padding=2 stride=1     \\ \hline
\rowcolor[HTML]{EFEFEF} 
ReLU           & -                                                \\ \hline
\rowcolor[HTML]{FFCCC9} 
POOL           & pool size {[}2, 2{]} stride=2                    \\ \hline
\rowcolor[HTML]{CBCEFB} 
CONV           & 64 filters 1 $\times$ 1, padding=0 stride=1      \\ \hline
\rowcolor[HTML]{EFEFEF} 
ReLU           & -                                                \\ \hline
Dropout        & p=0.5                                            \\ \hline
\rowcolor[HTML]{FBF1A2} 
FC             & 256 units                                        \\ \hline
Dropout        & p=0.5                                            \\ \hline
\rowcolor[HTML]{FBF1A2} 
FC             & \#Classes (=10) units                            \\ \hline
\end{tabular}
\caption{LeNet-2 architecture, used in Zhang et al. \parencite{zhang2015SVD}. }
\label{tab:lenet2}
\end{table}

subsection{CCNN}
Proposed by Poggi and Mattoccia \parencite{WCCNN} at BMVC 2016, CCNN is a single channel network capable of infering the confidence measure of a pixel just by taking as the only input cue its disparity map. CCNN outperformed state-of-the-art methods with margin peaks of 20\%, being the first method in this specific task to be based on deep learning. 
\newline 

The architecture of CNN is made of a single channel network that takes as input N×N patches, each one containing disparity values normalized between zero and one, represented by a 1×N×N tensor. 



The first part of this model is made of $\frac{N-1}{2}$ convolutional layers (with N equals to the patch size), each one followed by a Rectifier Linear Unit (ReLU).
Each convolutional layer contains F filters of size 3×3. No padding or stride is applied, making the final output of the convolutional layers, a F×1×1 tensor (each layer reduces the initial size N by 2 pixels), directly forwarded to the fully-connected part of the network deploying  two  layers,  made  of L neurons  each,  followed  by  ReLUs  (1).   The  final  layer collapses into a single neuron in charge of the regression. According to a common methodology usually deployed when dealing with deep architectures, the authors replaced  fully-connected layers with convolutional layers made of L 1x1 kernels.  This makes possible to train the network on image patches  as well as to compute a dense confidence map with a single forward of the full resolution image with a 0-padding of $\frac{N-1}{2}$ around it, keeping for the output the same w×h size of the input disparity map due to the absence of pooling operations or stride factors inside the convolutional layers. 

The CCNN architecture is described in figure \ref{fig:CCNN}. 


 \begin{figure}[h!]
 \centering
 \includegraphics[width=0.75\textwidth]{\path/CCNN.png} 
 \caption[CCNN architecture overview]{Architecture of CCNN to infer the CM from from the raw disparitymap.  It is a single channel network, designed for 9×9 image patches.  Four convolutional layers apply 64 overlapping kernels (stride equal to 1) of size 3×3. Two fully-connected layers made of 100 neurons each (i.e., 100 1×1 convolution kernels) lead to the final regression node.}.}
 \label{fig:CCNN}
\end{figure}


\subsection{Network-In-Network}
The Network-In-Network model (NIN) has been proposed in \citep{NIN} as a peculiar model which first leveraged on one-by-one convolution, explained in chapter \ref{Chapter3}. It features a fully-convolutional architecture with three main layers that can be decomposed, conv1, conv4 and conv7. 

The modified version, used also in Zhang et al \parencite{zhang2015SVD}, is depicted in figure \ref{fig:NIN}.
\begin{figure}[h!]
 \centering
 \includegraphics[width=0.65\textwidth]{\path/NIN.png} 
 \caption{Network in Network architecture. The low rank version is the one by Zhang et al\parencite{zhang2015SVD} }
 \label{fig:NIN}
\end{figure}

\subsection{Core strategy}
As previously mentioned, the experiments follow the micro-architecture pattern with the goal of testing the effectiveness of the TD-Block on both new architectures trained from scratch and pre-trained models. 

\subsubsection{TD-Block design}
As for the model design, different TD-block configurations have been tested. The base model is the one depicted in \ref{fig:td-block}. The middle convolution block can be either separable or regular spatial convolution, depending on the type of decomposition. \\
Since this is an arbitrary design choice, the following points were weighed the decision: 

\begin{itemize}
    \item since we are going to train from scratch, there is no particular reconstruction error constraint to recover.
    
    \item the goal is to find the most effective way to compress a CNN.
\end{itemize}

Therefore the type of convolution in the middle have always been the separable one suggested in section $4.3.1$ of chapter \ref{Chapter4}, proposed by Lebedev et al. \parencite{lebedev}. This configuration as a building block, represents an original idea that has not been fully explored, while being also the more aggressive compression. 

 \begin{figure}[h!]
 \centering
 \includegraphics[width=0.5\textwidth]{\path/TD-block.png} 
 \caption{The Tensor Decomposition (TD) block proposed in chapter \ref{Chapter4}. Variations of these block have been tried several times, with ReLU and batch normalization layers in between the three stages.}
 \label{fig:bigdata}
\end{figure}

Two main variations on the original TD-block were tried: 
\begin{enumerate}[(i)]
    \item adding a BatchNorm layer between each convolution; 
    
    \item adding ReLUs plus BatchNorm between each convolution. 
\end{enumerate}

This design scheme has been tested on LeNet-1 and CCNN. 

Note that other configuration are possible and will be explored in future work. 

\subsubsection{TD-Block compression}
For each model, compression has been tried in both the Tucker and CP way, as illustrated in chapter \ref{Chapter4}. 

Given that Tucker decomposition is more conservative, in the experiments most of the time, it represents a good trade-off for fast convergence and good approximation. 
\newline 

On the contrary, CP-decomposition is more aggressive and during the experiments it has been pushed by enforcing smaller ranks compared to Tucker, and thus getting a more compressed model. As indicated in the framework for tensor decomposition proposed in \ref{subsec:framework} of the previous chapter, sometimes VBMF estimates ranks that are too high, in the sense that it is possible to set a lower rank and then gamble on fine-tuning to recover (in higher time) the same accuracy.  

\subsubsection{Weight initialization}
As mentioned in chapter \ref{Chapter3}, weight initialization is important to get a faster convergence to the global minimum. The initialization is, in fact, together with the optimizer, the main tool that is able to change how the BackProp algorithm starts and ends through the loss.


After decomposing a tensor, the weights need to be adjusted to recover the accuracy; therefore, it is also possible to exploit the structure of the decomposition but initialize the weights in a different way. The former would assume that, since the TD-block is appropriate to substitute that specific layer, a uniform random initialization could also reach a global minimum after few iteration. 

\paragraph{Xavier}
Xavier initialization makes sure the weights are "just right" (not too large neither too small), keeping them in a reasonable range. Basically, Xavier init selects the adequate standard deviation in order to avoid the well-known issues of vanishing and exploding gradients. 

As suggested for future work by Kim et al. \parencite{Tucker-mobile}, Xavier init has been tried throughout the experiments. 


%--------------------------------------------------------------------
%	SECTION 2
%--------------------------------------------------------------------
\section{Experiment 1: TD-Design LeNet-1}
This experiment aims at discovering the effectiveness of the TD-block design. 


Training was done for 50 epochs, with Adam optimizer and a learning rate $\eta = 0.001$ decreasing by 0.1 every 20 epochs. In this way, the baseline model reached an accuracy of $74\%$.  

\subsection{TD-Block architecture}
The modified architecture of LeNet-1 has two main differences: (i) it converts the FC-layers to convolutional ones and (ii) substitutes each of the layer with its corresponding TD-block, with a CP-like configuration. 
\newline 

The full decomposed architecture has only 27K parameters, boasting a $45 \times$ compression ratio. The rank has been set differently for the first convolution layers (lower rank) and the last ones (higher rank), with values in the range $[10-40]$ Details about the architecture are reported in table \ref{tab:lenet1-cpd}.

% Please add the following required packages to your document preamble:
% \usepackage[table,xcdraw]{xcolor}
% If you use beamer only pass "xcolor=table" option, i.e. \documentclass[xcolor=table]{beamer}
\begin{table}[]
\centering
\begin{tabular}{|c|c|}
\hline
\textbf{LAYER} & \textbf{CHARACTERISTICS} \\ \hline
\rowcolor[HTML]{CBCEFB} 
TD-Block & $32 \times [1,1] + R \times 3 + 3 \times R + 32 \times [1,1] $ \\ \hline
\rowcolor[HTML]{EFEFEF} 
ReLU & - \\ \hline
\rowcolor[HTML]{CBCEFB} 
TD-Block & $32 \times [1,1] + R \times 3 + 3 \times R + 32 \times [1,1] $ \\ \hline
\rowcolor[HTML]{EFEFEF} 
ReLU & - \\ \hline
\rowcolor[HTML]{FFCCC9} 
POOL & pool size {[}2, 2{]} stride=2 \\ \hline
Dropout & p=0.25 \\ \hline
\rowcolor[HTML]{CBCEFB} 
TD-Block & $64 \times [1,1] + R \times 3 + 3 \times R + 64 \times [1,1] $ \\ \hline
\rowcolor[HTML]{EFEFEF} 
ReLU & - \\ \hline
\rowcolor[HTML]{CBCEFB} 
TD-Block & $64 \times [1,1] + R \times 3 + 3 \times R + 64 \times [1,1] $ \\ \hline
\rowcolor[HTML]{EFEFEF} 
ReLU & - \\ \hline
\rowcolor[HTML]{FFCCC9} 
POOL & pool size {[}2, 2{]} stride=2 \\ \hline
\rowcolor[HTML]{FFFFFF} 
Dropout & p=0.25 \\ \hline
\rowcolor[HTML]{FBF1A2} 
TD-Block & $64 \times [1,1] + R \times 6 + 6 \times R + 512 \times [1,1] $ \\ \hline
\rowcolor[HTML]{FBF1A2} 
TD-Block & $512 \times [1,1] + R \times 1 + 1 \times R + 10 \times [1,1] $ \\ \hline
\end{tabular}
\caption[LeNet-1 with TD-Block architecture]{Architecture of LeNet-1 with CP-like TD-blocks. Each TD-block has four convolution layers. The number of parameters is reported in the characteristics column. Note that, each TD-block has BatchNorm layers between each convolution. }
\label{tab:lenet1-cpd}
\end{table}

\subsection{Original vs. Decomposed}
Training with the same setup leads to the result shown in figure \ref{fig:baseline-TD}. 

\begin{figure}
    \subfloat{\label{sublable1}\includegraphics[width=.9\textwidth]{\plots/TD-vs-baseline.png} \\
    \subfloat{\label{sublable2}\includegraphics[width=.9\textwidth]{\plots/TD-vs-baseline-loss.png} 
    \caption{TD Architectures comparison for training accuracy and training loss.}
    \label{fig:TD-comparison}
\end{figure}

Clearly, the TD-block design outperforms the baseline by a significant margin, by reaching an accuracy of $82\%$ against the original $74\%$ of the LeNet-1 model. By looking at the training loss we can see that it also converges faster to a better minimum.  


\subsection{TD configurations comparison}
To avoid confusion the CP-like configuration of the TD-block will have a CP-prefix. That aside, three main configuration of the TD-block were tested:

\begin{enumerate}
    \item CP-BN: \texttt{CONV-BN-CONV-BN-CONV-BN-CONV-BN}
    
    \item CP-BN-RELU: \texttt{CONV-BN-RELU-CONV-BN-RELU-CONV-BN-RELU-CONV-BN-RELU}
    
    \item CP-BN-Xavier: same as the CP-BN but with Xavier initialization. 
\end{enumerate}

As can be notice, the original configuration from Lebedev et al \parencite{lebedev} is missing. The main issue is that in the proposed all TD-block architecture \ref{tab:lenet1-cpd}, the number of convolutional layers skyrockets from 4 to 24 ($\times$ 4, plus the conversion of the FC layers). Therefore, it probably incurs in the vanishing gradient problem, as the training is too slow to converge. 
\newline 

The comparison between the three proposed configurations is shown in figure. \ref{fig:TD-comparison}


\begin{figure}
    \subfloat{\includegraphics[width=1\textwidth]{\plots/TD-architecture-acc.png} \\
    \subfloat{\includegraphics[width=1\textwidth]{\plots/TD-architecture-loss.png} 
    \caption{TD Architectures comparison for training accuracy and training loss.}
    \label{fig:TD-comparison}
\end{figure}

\paragraph{RELU effect}: 
As can be seen, the RELU configuration performs noticeably worse than the other two, which are closely comparable. 
As also found by F. Chollet in the context of depthwise separable convolution\parencite{chollet}, the absence of non-linearity leads to both faster convergence and better final performance. According to the author, it may be that the depth of the intermediate feature spaces on which spatial convolutions are applied is critical to the usefulness of the non-linearity: for feature spaces with many channels the non linearity is useful to enhance the most important ones, but for shallow ones - the CP-like architecture employs a 1-to-1 channel connection - it becomes possibly harmful, due to a loss of information. 

In other words, when the information contained in the layers is redundant non-linearities helps to extract the important features and squash away the rest; on the other side, the filters in the TD- block are so few that are "all important" and RELU would cut away important information. 

\paragraph{On Xavier init}: 
Surprisingly, Xavier initialization did not improve the overall performance of the network. At testing time the two configurations were similar with the regular one getting an $82\%$ and the xavier one an $81\%$ of accuracy. This may be due to the heavy batch normalization use amidst the convolutional layers that already helps convergence a lot. \\



\subsection{Overall} 
The TD architecture have proven to converge faster and to be able to generalize better than the baseline model, while having only $2\%$ of the original number of parameters. This leads to few observations: 
    \begin{itemize}
        \item the very low number of parameters acts as a form of \emph{regularization}, preventing the model to be overly complex, thus avoiding over-fitting. 
        
        \item There is a very high percentage of redundancy between the channels of the CNN. This is in line with what has been explained in chapter 3 about network design. 
        
        \item There is an exploitable correlation also between the 9 (in this case) values of each kernel. This is a good hint for future applications of separable convolution (see chapter 3, section \ref{subsec:separable}), that will provide an even more effective compression when the kernel size is 5 of higher. \\
        Differently from the previous point, this architecture has not been exploited yet in literature. 
        
        \item FC layers can be converted and then decomposed as well, going from 1.18 millions down to $\approx$8 thousands. The latter, of course, brings most of the compression ratio. 
    \end{itemize}

Table \ref{tab:td-conf} reports a summary of the results. 
\newline 

% Please add the following required packages to your document preamble:
% \usepackage{booktabs}
\begin{table}[]
\centering
\begin{tabular}{@{}cccc@{}}
\toprule
Model & Parameters & Test Accuracy & $\mathcal{C}_r$ \\ \midrule
\multicolumn{1}{|c|}{LeNet-1 (baseline)} & \multicolumn{1}{c|}{$\approx$ 1.2 Million} & \multicolumn{1}{c|}{74\%} & \multicolumn{1}{c|}{-} \\ \midrule
\multicolumn{1}{|c|}{CP-BN} & \multicolumn{1}{c|}{$\approx$ 27K} & \multicolumn{1}{c|}{82\% (+8\%)} & \multicolumn{1}{c|}{45 $\times$} \\ \midrule
\multicolumn{1}{|c|}{CP-BN-Xavier} & \multicolumn{1}{c|}{$\approx$27K} & \multicolumn{1}{c|}{81\% (+7\%)} & \multicolumn{1}{c|}{45 $\times$} \\ \midrule
\multicolumn{1}{|c|}{CP-BN-RELU} & \multicolumn{1}{c|}{$\approx$ 27K} & \multicolumn{1}{c|}{77\% (+3\%)} & \multicolumn{1}{c|}{45 $\times$} \\ \bottomrule
\end{tabular}
\caption{Accuracy and parameters for the baseline and the different configurations of TD-block models.}
\label{tab:td-conf}
\end{table}

This is a promising result and says something about how many parameters in a CNN are actually necessary. Of course, it also depends on the complexity of the dataset, nevertheless a good approximation seems to plausible even for larger models. 


\subsection{Future ideas}
With regard to the loss of information caused by RELU, an idea may be suggested by deep residual networks \ref{subsec:resnet}. By summing up the original input of the TD block to the output, as in Resnet, the loss of information could be recovered. Furthermore, it may even improve overall accuracy. Therefore, a residual learning configuration of the TD-block could be investigated in future work. 

%--------------------------------------------------------------------
%	SECTION 3
%--------------------------------------------------------------------
\section{Experiment 2: TD-Compression LeNet-Conv}
In this experiment we are going to actually compress a similar architecture to one trained from scratch in the previous experiment. During the experiments, few questions came out like which layer should be compressed before; in which direction should the compression go and such. The results of this compression could be helpful to setting up guidelines for the future. 


\paragraph{Decomposition pipeline}
Experiments have shown there is no much difference between the order of the decomposition, as long as the decomposition are computed accurately, that is. \\

Basically, four strategies can be outlined: 
\begin{enumerate}
    \item \emph{ordered}: from first layer to the last; 
    
    \item \emph{reversed}: from last layer to the first; 
    
    \item \emph{larger first}: the first layer to be compressed is the one with the highest number of parameter. This can be tricky, but once it is optimized, the other layers should not cause many troubles even with aggressive compression rates. \\
    The downside of this method is in the pipeline strategy itself. If the compression is performed on each layer subsequently and in an automatic manner (e.g. executed by an automated script) without a supervision, it could result in a bad compression. This is because the largest layer is usually the hardest to be compressed accurately and if the approximation is too harsh, it could be impossible to recover. Consequently, the error propagates on the subsequent compressions as well, resulting, as said, in a drop in accuracy.  
    

    \item \emph{smaller first} layers get compressed in ascending order of number of parameters. This is probably the safest bet to achieve a good compression. However, it also hides a pitfall. For instance, let A,B and C three convolutional layers with 100K, 700K and 1.7M parameters respectively. Let's say the compression rate has been kept low on A and B in order to do a very aggressive compression only on C (that will contribute more on the overall size), and let C be particularly sensitive to the overall accuracy of the network. In this latter scenario, we could find that the only good approximation of C is a very soft one and thus ending up with very few parameters less than the original model. 
\end{enumerate}

Results of CP-decomposition in both ordered and reversed way are shown in plots \ref{fig:order-comparison}. As it can be seen, results are almost stackable. 

\begin{figure}
    \subfloat{\label{sublable1}\includegraphics[width=1\textwidth]{\plots/order-comparison-acc.png} \\
    \subfloat{\label{sublable2}\includegraphics[width=1\textwidth]{\plots/order-comparison-loss.png} 
    \caption{CP-decomposition: the order of the layers decomposition does not affect overall performance.}
    \label{fig:order-comparison}
\end{figure}



\subsection{CP}
Contrary to what usually reported in literature\parencite{lebedev}, Parafac decomposition was comparable to Tucker stability for this experiment. Especially on smaller layers, CP converges very fast (\ref{fig:order-comparison}), as Tucker usually do. 
\newline 

When dealing with the FC2Conv layer with more than a million parameters, the training is not as smooth but it still converges to a global minimum (loss is 0.19), improving the baseline results by $1\%$ while compressing the whole model by boasting a 29$\times$ compression, with a peak compression rate $\mathcal{C}_r$ of $40\times$ on the larger layer.  

A recap of CP results is in table \ref{tab:allconv-cp-results}. Note the difference between the suggested VBMF rank for the last layer (R=170) and the rank imposed by selecting the desired compression ratio (R=50). 


\begin{table}[]
\begin{center}
    

\begin{tabular}{@{}l||lllll@{}}
 \hline
 \multicolumn{6}{|c|}{\textbf{Parafac Decomposition}} \\
 \hline
\toprule
\textbf{Layer}                                                                             & \textbf{Start}                                                                   & \textbf{Fine-tuned}                                                                           & \textbf{Params}                & \textbf{Compres.}               & \textbf{$\mathcal{C}_r$}                      \\ \midrule
\multicolumn{1}{|l|}{CONV4}                                                                & \multicolumn{1}{l|}{\begin{tabular}[c]{@{}l@{}}Acc: 79\%\\ Loss : 0.193\end{tabular}} & \multicolumn{1}{l|}{\begin{tabular}[c]{@{}l@{}}Acc: 77\%\\ Loss: 0.023\end{tabular}}          & \multicolumn{1}{l|}{36928}            & \multicolumn{1}{l|}{2878}           & \multicolumn{1}{l|}{\textbf{13x}}  \\ \midrule
\multicolumn{1}{|l|}{CONV(4+3)}                                                            & \multicolumn{1}{l|}{\begin{tabular}[c]{@{}l@{}}Acc: 77\%\\ Loss: 0.082\end{tabular}}  & \multicolumn{1}{l|}{\begin{tabular}[c]{@{}l@{}}Acc: 77\%\\ Loss: 0.018\end{tabular}}          & \multicolumn{1}{l|}{18496}            & \multicolumn{1}{l|}{2206}           & \multicolumn{1}{l|}{\textbf{8.5x}} \\ \midrule
\multicolumn{1}{|l|}{CONV(4+3+2)}                                                          & \multicolumn{1}{l|}{\begin{tabular}[c]{@{}l@{}}Acc: 76\%\\ Loss: 0.300\end{tabular}}  & \multicolumn{1}{l|}{\begin{tabular}[c]{@{}l@{}}Acc: 77\%\\ Loss: 0.033\end{tabular}}          & \multicolumn{1}{l|}{9248}             & \multicolumn{1}{l|}{732}            & \multicolumn{1}{l|}{\textbf{13x}}  \\ \midrule
\multicolumn{1}{|l|}{CONV(4+3+2+1)}                                                        & \multicolumn{1}{l|}{\begin{tabular}[c]{@{}l@{}}Acc: 77\%\\ Loss: 0.081\end{tabular}}  & \multicolumn{1}{l|}{\begin{tabular}[c]{@{}l@{}}Acc: 77\%\\ Loss: 0.033\end{tabular}}          & \multicolumn{1}{l|}{896}              & \multicolumn{1}{l|}{442}            & \multicolumn{1}{l|}{\textbf{2x}}   \\ \midrule
\multicolumn{1}{|l|}{\begin{tabular}[c]{@{}l@{}}CONV2FC1\\ Rank=170\end{tabular}}          & \multicolumn{1}{l|}{Loss: 0.738}                                                      & \multicolumn{1}{l|}{\begin{tabular}[c]{@{}l@{}}Acc: 80\%\\ Loss: 0.195\end{tabular}}          & \multicolumn{1}{l|}{1180160}          & \multicolumn{1}{l|}{100472}         & \multicolumn{1}{l|}{\textbf{12x}}  \\ \midrule
\multicolumn{1}{|l|}{\textbf{\begin{tabular}[c]{@{}l@{}}CONV2FC1 \\ Rank=50\end{tabular}}} & \multicolumn{1}{l|}{\textbf{Loss: 0.981}}                                             & \multicolumn{1}{l|}{\textbf{\begin{tabular}[c]{@{}l@{}}Acc: 80\%\\ Loss: 0.321\end{tabular}}} & \multicolumn{1}{l|}{\textbf{1180160}} & \multicolumn{1}{l|}{\textbf{29912}} & \multicolumn{1}{l|}{\textbf{40x}}  \\ \midrule
\textbf{Overall}                                                                           & \textbf{Acc: 79\%}                                                                    & \textbf{Acc: 80\%}                                                                            & \textbf{1252480}                      & \textbf{42922}                      & \textbf{29x}                       \\ \bottomrule
\end{tabular}

\caption{A comprehensive summary of the CP-decomposition on LeNet-Conv. Start indicates the testing accuracy and training loss before decomposing the layer, while the fine-tuned column shows the same metrics after the decomposion+fine-tuning step. In bold the most interesting results. }
\label{tab:allconv-cp-results}
\end{center}
\end{table}

\subsection{Tucker}
Tucker achieved similar results but with a weaker compression, probably due to the conservative rank estimation of VBMF. Enforcing of Tucker ranks by selecting a desired compression ratio has been also implemented in the 'decomposer' class but did not achieve the same accuracy of CP at this time. 
\newline 

Loss and training accuracy can be observed in figure \ref{fig:allconv:tucker}. Training have been done with early stopping, that is why the first layers decomposition stopped around epoch 10, when they actually reach the minimum. The largest layer (FC2CONV) has more than 1 million weights, that is why its convergence takes much more. Nevertheless , the decomposition seems to bring new benefits to the generalization capacity of the model, as reported in the previous experiment. In fact, the overall accuracy on the test set at the end of the decopmosition is of 77\%. 
 % Overall compression rates are instead reported in table \ref{tab:allconv-tucker}. 

\begin{figure}
\centering
    \subfloat{\label{sublable1}\includegraphics[width=.7\textwidth]{\plots/tucker2-acc-allconv.png} \\
    \subfloat{\label{sublable2}\includegraphics[width=.7\textwidth]{\plots/tucker2-loss-allconv.png} 
    \caption{Tucker decomposition: most of the layers reach the global minimum very fast. For obvious reasons, the largest layer comprised of 1.18 million parameters takes more time to converge.}
    \label{fig:allconv-tucker}
\end{figure}

\paragraph{Mixed Decomposition}: interestengly, it is also possible to mix Tucker and CP in a gradually more aggressive decomposition. This is perfomed by first applying Tucker on the full convolutional layer, and then applying CP decomposition on the central block of the Tucker decomposition. The latter, in fact, features a squared regular spatial convolution that can be further accounted by CP decomposition. This scheme was tried only after the last convergence of Tucker and it managed to reduce the size of the layer by another 5X while showing \emph{no} accuracy drop (i.e. 77\% as before). 


\subsection{Xavier}
It has also tested the CP-Xavier method: applying CP-decomposition architecture first (without computing the actual weights) and initializing with Xavier uniform distribution. %( \texttt{torch.nn.init.xavier_init} for pytorch, and \texttt{tf.contrib.layers.xavier_initializer} for tensorflow). 
\newline 

Training loss is shown in \ref{fig:allconv-xavier}. Xavier converges as fast as CP in this case but being much smoother, and while fluctuating a bit on testing (77-74-75-77) at the end achieves an optimal accuracy of $79\%$ as Tucker. Hence, this means there is no accuracy drop. 


 \begin{figure}[h!]
 \centering
 \includegraphics[width=1.0\textwidth]{\plots/allconv-xavier-loss.png} 
 \caption{CP decomposition with Xavier initialization. Convergence as fast as the other methods, but in a smoother way, confirming the benefit of this type of initialization. }
 \label{fig:allconv-xavier}
\end{figure}


\subsection{Overall}
This experiment have shown how the order of the decomposition does not affect the final result as long as the approximation can be recovered by fine-tuning for 10 to 50 epochs. Tucker and CP-Xavier had achieved similar results on both compression and accuracy while CP decomposition managed to improve accuracy while having a whopping $\approx 30$ less parameters. 

Overall, all decomposition methods showed promising results. Also, mixed decomposition could represent a combination of Tucker conservative decomposition and CP aggressive one into one pipeline. 


%--------------------------------------------------------------------
%	SECTION 4
%--------------------------------------------------------------------

\section{Experiment 3: TD-Compression LeNet-2}
This second LeNet configuration has more dense layers that can be harder to compress but that could also lead to a higher compression rate. In this experiment we will also see how Xavier init compares with CP decomposition with different results compared to the previous experiment. The training of the baseline model have been done without data augmentation (differently from \parencite{zhang2015SVD}) and the baseline model reached an accuracy of 74\% in 50 epochs. That must be kept in mind for comparison results. 

The only layers that have been decomposed are the first three convolutions, in reversed order (conv3-conv2-conv1). However, it must be noted that the last fully-connected layers can be decomposed by applying SVD on its matrices, as explained in chapter \ref{Chapter4}, section \ref{subsec:svd-fc}. 

\subsection{CP}
CP decomposition performs very well on this network too, confirming the results of the previous experiment. 
The convergence of all the three decomposition is shown in \ref{fig:zhang-cp-loss}. Interestingly, the decomposition of all three layers converges better than the one of only the second and the one. This may be due to some kind of cross-layer redundancy that can be improved by decomposition only on that specific junction of layers. 

\begin{figure}[h!]
 \centering
 \includegraphics[width=1.0\textwidth]{\plots/zhang-cp-loss.png} 
 \caption{CP decomposition on LeNet-2. }
 \label{fig:zhang-cp-loss}
\end{figure}


By compressing the whole model of 4.3$\times$ and the largest layer of over 30$\times$ it is able to achieve an accuracy on the test set of $82\%$, which a +3\% improvement on baseline model. Remarkably, when enforcing a compression rate of 52X on conv3, as reported in \parencite{zhang2015SVD}, CP managed to achieve a 77\% accuracy on the test set, which is still better then baseline. 

More details are shown in table \ref{tab:zhang-cp}. 

% Please add the following required packages to your document preamble:
% \usepackage{booktabs}
\begin{table}[]
\centering
\caption{CP-overall compression results}
\label{tab:zhang-cp}
\begin{tabular}{@{}llll@{}}
\toprule
Layer                                                                               & Accuracy                                  & $\mathcal{C}_r$                   & Speed-up                           \\ \midrule
\multicolumn{1}{|l|}{Conv1}                                                         & \multicolumn{1}{l|}{80\% (+6\%)}          & \multicolumn{1}{l|}{4X}           & \multicolumn{1}{l|}{2.3X}          \\ \midrule
\multicolumn{1}{|l|}{Conv2}                                                         & \multicolumn{1}{l|}{82\% (+8\%)}          & \multicolumn{1}{l|}{31X}          & \multicolumn{1}{l|}{1.7X}          \\ \midrule
\multicolumn{1}{|l|}{\begin{tabular}[c]{@{}l@{}}Conv3\\ R=VBMF\end{tabular}}        & \multicolumn{1}{l|}{80\% (+6\%)}          & \multicolumn{1}{l|}{25X}          & \multicolumn{1}{l|}{1.2X}          \\ \midrule
\multicolumn{1}{|l|}{\textbf{\begin{tabular}[c]{@{}l@{}}Conv3\\ R=39\end{tabular}}} & \multicolumn{1}{l|}{\textbf{77\% (+3\%)}} & \multicolumn{1}{l|}{\textbf{52X}} & \multicolumn{1}{l|}{\textbf{1.5X}} \\ \midrule
\textbf{Overall}                                                                    & \textbf{81\% (+7\%)}                      & \textbf{4.5X}                     & \textbf{3.25X}                     \\ \bottomrule
\end{tabular}
\end{table}
 

\subsection{Tucker}
Tucker decomposition improved the accuracy as well, accounting for +6\%. The overall compression ratio is much smaller than CP when both uses VBMF rank estimation. However, Tucker proved to be more stable and training took less iterations than the CP correspectives. Sometimes 5 epochs against 30-40 required by CP. Furthermore, when selecting a very high compression rate, like 63X, on the third convolutional layer, the training managed to converge as well, to a 76\%. The latter is an improving on baseline while providing an aggressive compression as well. 

% Please add the following required packages to your document preamble:
% \usepackage{booktabs}
\begin{table}[]
\centering
\caption{Tucker overall compression results.}
\label{tab:zhang-tucker}
\begin{tabular}{@{}llll@{}}
\toprule
Layer                                                                                      & Accuracy                                  & $\mathcal{C}_r$                   & Speed-up                           \\ \midrule
\multicolumn{1}{|l|}{Conv1}                                                                & \multicolumn{1}{l|}{75\% (+1\%)}          & \multicolumn{1}{l|}{0.77X}        & \multicolumn{1}{l|}{2.7X}          \\ \midrule
\multicolumn{1}{|l|}{Conv2}                                                                & \multicolumn{1}{l|}{79\% (+8\%)}          & \multicolumn{1}{l|}{6.3X}         & \multicolumn{1}{l|}{1.7X}          \\ \midrule
\multicolumn{1}{|l|}{\begin{tabular}[c]{@{}l@{}}Conv3\\ R3,R4=VBMF\end{tabular}}           & \multicolumn{1}{l|}{80\% (+6\%)}          & \multicolumn{1}{l|}{14.5X}        & \multicolumn{1}{l|}{1.2X}          \\ \midrule
\multicolumn{1}{|l|}{\textbf{\begin{tabular}[c]{@{}l@{}}Conv3\\ R3=6, R4=41\end{tabular}}} & \multicolumn{1}{l|}{\textbf{76\% (+2\%)}} & \multicolumn{1}{l|}{\textbf{63X}} & \multicolumn{1}{l|}{\textbf{1.5X}} \\ \midrule
\textbf{Overall}                                                                           & \textbf{80\% (+6\%)}                      & \textbf{3.5X}                     & \textbf{3.25X}                     \\ \bottomrule
\end{tabular}
\end{table}



\subsection{Xavier comparison}
After several experiments, an interesting point about Xavier init came out. Depending on the quality of the decomposition, CP can converge faster and to a better minimum. This may be due to initial solution, ironically, of the tensor decomposition algorithm itself. An example is shown in figure \ref{fig:xavier-vs-cp}, where two different CP decompositions, A and B, for the same layer converge in different ways; xavier init trajectory is almost identical to B, but inferior to the better A-CP decomposition.
\
On the other hand, Tucker decomposition is more stable and converge much faster than a random xavier init. As shown in figure \ref{fig:xavier-vs-tucker}.  


\begin{figure}[h!]
 \centering
 \includegraphics[width=1.0\textwidth]{\plots/xavier-vs-cp-loss-conv3.png} 
 \caption{CP vs. Xavier init comparison for different decomposition of the same layer.}
 \label{fig:xavier-vs-cp}
\end{figure}

\begin{figure}[h!]
 \centering
 \includegraphics[width=1.0\textwidth]{\plots/xavier-vs-tucker.png} 
 \caption{Tucker vs. Xavier init comparison for conv2 layer.}
 \label{fig:xavier-vs-tucker}
\end{figure}


\subsection{Overall}
The reported results seem to improve upon the one reported by Zhang et al \parencite{zhang2015SVD}. However, the models may be not fully comparable, due to the lack of data augmentation and the different frameworks. \\
Noticeably, even with very high compression rates (> 50x) both decomposition techniques managed to achieve an improvement over baseline, with CP getting higher accuracies in the case of such aggressive compression ratios. 


%--------------------------------------------------------------------
%	SECTION 5
%--------------------------------------------------------------------
\section{Experiment 4: TD-Compression NIN}
NIN models introduced in the previous sections, is able to get to 86\% on CIFAR-10 without data augmentation. This specific model can be decomposed only on layer 1, 4 and 7. The other layers are all one-by-one convolutions and thus not suitable for tensor decomposition. 

\subsection{CP}
CP-decomposition achieved a 5X compression ratio while improving the overall accuracy by 1 point. The training curve in this case was also smooth. Training results are shown in \ref{fig:NIN-cp} while details about compression ratio and overall speedup are reported in table \ref{tab:NIN-cp}.

\begin{figure}
\centering
    \subfloat{\label{}\includegraphics[width=1\textwidth]{\plots/NIN-cp-acc.png} \\
    \subfloat{\label{}\includegraphics[width=1\textwidth]{\plots/NIN-cp-loss.png} 
    \caption{Training metrics for CP-decomposition of NIN model.}
    \label{fig:NIN-cp}
\end{figure}

% Please add the following required packages to your document preamble:
% \usepackage{booktabs}
\begin{table}[]
\centering
\caption{CP overall results on NIN.}
\label{tab:NIN-cp}
\begin{tabular}{@{}llll@{}}
\toprule
Layer                                                                       & Accuracy                         & $\mathcal{C}_r$            & Speed-up                  \\ \midrule
\multicolumn{1}{|l|}{\begin{tabular}[c]{@{}l@{}}Conv1\\ R=14\end{tabular}}  & \multicolumn{1}{l|}{86\% (+0\%)} & \multicolumn{1}{l|}{4.7X}  & \multicolumn{1}{l|}{2.0X} \\ \midrule
\multicolumn{1}{|l|}{\begin{tabular}[c]{@{}l@{}}Conv7\\ R=64\end{tabular}}  & \multicolumn{1}{l|}{87\% (+1\%)} & \multicolumn{1}{l|}{12.8X} & \multicolumn{1}{l|}{1.7X} \\ \midrule
\multicolumn{1}{|l|}{\begin{tabular}[c]{@{}l@{}}Conv4\\ R=100\end{tabular}} & \multicolumn{1}{l|}{86\% (+0\%)} & \multicolumn{1}{l|}{15X}   & \multicolumn{1}{l|}{1.2X} \\ \midrule
\textbf{Overall}                                                            & \textbf{87\% (+1\%)}             & \textbf{$\sim$5X}          & \textbf{2.5X}             \\ \bottomrule
\end{tabular}
\end{table}

\subsection{Tucker}
Tucker has proven again to be stable and consistent among the decompositiong. Choosing the ranks with VBMF, it compressed NIN by 2x. Training metrics are shown in figure \ref{fig:NIN-tucker}.  More details are in table \ref{tab:NIN-tucker}.

\begin{figure}[h!]
 \centering
 \includegraphics[width=1.0\textwidth]{\plots/NIN-tucker.png} 
 \caption{Tucker compression on NIN's three decomopsable layers. Note how bad Xavier init performes on one of them.}
 \label{fig:NIN-tucker}
\end{figure}

% Please add the following required packages to your document preamble:
% \usepackage{booktabs}
\begin{table}[]
\centering
\caption{Tucker compression results on NIN.}
\label{tab:NIN-tucker}
\begin{tabular}{@{}llll@{}}
\toprule
Layer                                                                            & Accuracy                         & $\mathcal{C}_r$           & Speed-up                  \\ \midrule
\multicolumn{1}{|l|}{Conv1}                                                      & \multicolumn{1}{l|}{86\% (+0\%)} & \multicolumn{1}{l|}{1X}   & \multicolumn{1}{l|}{1.4X} \\ \midrule
\multicolumn{1}{|l|}{Conv7}                                                      & \multicolumn{1}{l|}{86\% (+0\%)} & \multicolumn{1}{l|}{6.7X} & \multicolumn{1}{l|}{1.7X} \\ \midrule
\multicolumn{1}{|l|}{\begin{tabular}[c]{@{}l@{}}Conv3\\ R3,R4=VBMF\end{tabular}} & \multicolumn{1}{l|}{86\% (+0\%)} & \multicolumn{1}{l|}{7.5X} & \multicolumn{1}{l|}{1.2X} \\ \midrule
\textbf{Overall}                                                                 & \textbf{86\% (+0\%)}             & \textbf{$\sim$2X}         & \textbf{2.5X}             \\ \bottomrule
\end{tabular}
\end{table}

\subsection{Overall}
With this model CP decomposition got better results, both in compression and in final accuracy. Both methods have shown no drop in accuracy compared to the baseline NIN. 

--------------------------------------------------------------------
%	SECTION 5
%--------------------------------------------------------------------
\section{Experiment 5: TD-Design CCNN}
With respect to real-time application, an interesting scenario is represented by 3D stereo vision. Stereo vision is a popular technique to infer depth from two or more images. In this field, confidence measures aim at detecting uncertain disparity assignments. From the observation that recurrent local patterns which occurs in disparity maps can tell a correct assignment from a wrong one, CCNN models the confidence formulation as a regression problem by analyzing disparity maps generated by a stereo vision system. 
\newline 

CCNN has only $\approx 128K$ parameters and generats a full confidence map in only 630 ms, thanks to its fully convolutional architecture. Being already fast and small,  CCNN represents a challenging testbed for tensor decomposition design. Nevertheless, the experiments shows that it can be improved by applying CP-like tensor decomposition blocks.  

\subsection{TD-Block architecture}
As it is already a fully convolutional architecture, it is relatively easy to re-design CCNN with the TD-block architecture described in section \ref{sec:TD-design}. Thus, the so modeled architecture will feature four TD-blocks made of 4 convolutional layers each, but, differently from the design of experiment 1 \ref{sec:exp1} \emph{without} batch normalization layers in between. 
\newline 

Two different models were tested:  
\begin{enumerate}
    \item CCNN with 4 TD-blocks, one per each convolution layer and the rest of the model unchanged: \texttt{TD-RELU-TD-RELU-TD-RELU-TD-RELU-FC1-FC2-REGRESSION_HEAD}
    
    \item CCNN with 4 TD-blocks plus a further decomposition of the FC-Conv layers into a combination of two smaller layers, according to an approximation inspired by singular value decomposition (SVD) \ref{subsec:SVD}.
    
\end{enumerate}

For each of the TD-block the compression ratio gain $\mathcal{C}_r$ is given by $$\mathcal{C}_r = \frac{(STd^2}{R(S+2d+T)}$$, 
where S, T stand for input and output maps dimension and $d$ for the kernel size, respectively and R is the decomposition rank. 

As for the SVD decomposition scheme, the idea is depicted as follows: 
\\
Given a FC layer of size $A \times B$, it is possible to divide it into two smaller layers of size $A \times R$ and $R \times B$ respectively. In this way the complexity goes from $S\cdot T$ to $R(S+T)$. Hence, a complexity advantage is guaranteed as long as $R < \frac{S\cdot T}{S+T} $. 

For both schemes, the rank R was set to be equal to 20, resulting in 36K parameters for the first decomposed model and 27K thousand for the second. 

\subsection{Training}
The network was trained for 14 epochs on $9 \times 9$ patches of the first 20 images of the KITTI stereo dataset, providing approximately 2.6 million samples.\\
The loss function to minimize is the Binary Cross Entropy (BCE) between the output $o$ and a label $t$ on each sample $i$ of the mini-batch. 


\section{Evaluation & Results}
Evaluation assessment was done by ROC curve analysis, which is commonly used when dealing with confidence measures. After ROC curves are depicted for each image, the Area Under The Curve is then used to evaluate the capability of the confidence measure to distinguish correct disparity assignments from erroneous ones, with respect to the optimatl solution. 

For the two models described above, the evaluation gives optimal results. In fact, both architectures outperforms the baseline as reported in table \ref{tab:CCNN}. Remarkably, the smaller architecture was the best scorer. 

\begin{table}[]
\centering
\begin{tabular}{|c|c|c|}
\hline
MODEL                & PARAMS       & AUC (Test Set)  \\ \hline
CCNN (baseline)      & 128K         & 0.1222          \\ \hline
TD Design 1          & 36K          & 0.1192          \\ \hline
\textbf{TD Design 2} & \textbf{27K} & \textbf{0.1186} \\ \hline
\textit{Optimal}     & -            & \textbf{0.1073} \\ \hline
\end{tabular}
\caption{AUC evaluation on 174 images of the KITTI stereo benchmark dataset. The smallest model outperforms the baseline. }
\label{tab:CCNN}
\end{table}

For each model, the AUC comparison with the optimal solution is depicted in \ref{fig:AUC}. We can see that the models go very close to the optimal solution.

\begin{figure}
\centering
    \subfloat{\label{}\includegraphics[width=1\textwidth]{\plots/AUC-27K.png} \\
    \subfloat{\label{}\includegraphics[width=1\textwidth]{\plots/AUC-36K.png} 
    \caption{Area Under the Curve (AUC) computed per each test image for the two decomposed models (in green) and the optimal solution (in orange).}
    \label{fig:AUC}
\end{figure}


\subsection{Discussion}
These last results are surprising, as the baseline had already very few parameters compared to typical CNN architectures. This suggests that, after all, some sorta of redundancy must is present in both the kernels and the channels. It may due to the fact that CCNN process overlapping $9 \times 9$ patches that could contain redundant information that can be exploited by cross-channel pooling, as performed by the TD-block design. 

Moreover, the fewer parameters may act as another form of regularization and thus enhances the generalization capabilities of the network. Further investigations are needed to deeply understand the decomposed architecture's benefits, such as applying a different kind of channel pooling (e.g. varying the stride) and then test if the same results still applies. This is left as future work. 



\section{Analysis of the results}
Most experiments have proven that tensor decomposition is a good candidate for the task it has been chosen for. For both Tucker and CP, different type of networks have never shown drop in accuracy. Xavier initialization however, even if sometimes beneficial and easier to implement, did not achieve good results on NIN and LeNet-2 models, compared to an accurate decomposition, that is. 
\newline

Tucker has been generally more stable than CP, showing faster convergence and thus being more reliable. However, CP decomposition has almost always been a better fit to perform an aggressive decomposition. The latter, though, needs more iterations to converge and thus may not be always feasible to perform. A curios pattern regard that of mixed decompositions that gradually perform conservative and aggressive decomposition, towards which more investigation is needed. 
\newline 

As for the objectives given in chapter 4 with respect to the TD-block, the expectations have been met in both model compression and model design. This TD-block pattern, managed to improve the accuracy of LeNet-1 by 8\% despite having 45x less parameters. Moreover, it achieved a remarakable result on the CCNN testbed, improving the baseline - which is state-of-the-art - by a fair margin, while having 6x less parameters. 


\begin{comment}
\begin{center}
\begin{table}[h!]

\centering
\caption{My caption}
\label{my-label}
\begin{tabular}{|l||l|l|l|l|l|}
 \hline
 \multicolumn{6}{|c|}{\textbf{Variabili}} \\
 \hline
\textbf{Layer}                                                       & \textbf{Start}                                              & \textbf{Fine-tuned}                                                      & \textbf{Params Before} & \textbf{Params After} & \textbf{Impr} \\ \hline
CONV4                                                                & \begin{tabular}[c]{@{}l@{}}Acc: 79\%\\ Loss : 0.193\end{tabular} & \begin{tabular}[c]{@{}l@{}}Acc: 77\%\\ Loss: 0.023\end{tabular}          & 36928                  & 2878                  & \textbf{13x}  \\ \hline
CONV(4+3)                                                            & \begin{tabular}[c]{@{}l@{}}Acc: 77\%\\ Loss: 0.082\end{tabular}  & \begin{tabular}[c]{@{}l@{}}Acc: 77\%\\ Loss: 0.018\end{tabular}          & 18496                  & 2206                  & \textbf{8.5x} \\ \hline
CONV(4+3+2)                                                          & \begin{tabular}[c]{@{}l@{}}Acc: 76\%\\ Loss: 0.300\end{tabular}  & \begin{tabular}[c]{@{}l@{}}Acc: 77\%\\ Loss: 0.033\end{tabular}          & 9248                   & 732                   & \textbf{13x}  \\ \hline
CONV(4+3+2+1)                                                        & \begin{tabular}[c]{@{}l@{}}Acc: 77\%\\ Loss: 0.081\end{tabular}  & \begin{tabular}[c]{@{}l@{}}Acc: 77\%\\ Loss: 0.033\end{tabular}          & 896                    & 442                   & \textbf{2x}   \\ \hline
\begin{tabular}[c]{@{}l@{}}CONV2FC1\\ Rank=170\end{tabular}          & Loss: 0.738  R=170                                               & \begin{tabular}[c]{@{}l@{}}Acc: 80\%\\ Loss: 0.195\end{tabular}          & 1180160                & 100472                & \textbf{12x}  \\ \hline
\textbf{\begin{tabular}[c]{@{}l@{}}CONV2FC1 \\ Rank=50\end{tabular}} & \textbf{Loss: 0.981,R=50}                                        & \textbf{\begin{tabular}[c]{@{}l@{}}Acc: 80\%\\ Loss: 0.321\end{tabular}} & \textbf{1180160}       & \textbf{29912}        & \textbf{40x}  \\ \hline
\textbf{Overall}                                                     & \textbf{Acc: 79\%}                                               & \textbf{Acc: 80\%}                                                       & \textbf{1252480}       & \textbf{42922}        & 29x           \\ \hline
\end{tabular}

\end{table}
\end{center}
\end{comment}
 
% Chapter Template

\chapter{Conclusioni} % Main c




% \chapter{Conclusioni} 
Questa attività progettuale ha permesso sia di testare sul campo che di fornire un'opportunità di studio ed analisi ad ampio spettro delle reti neurali artificiali, che hanno fatto tanto parlare di sé negli ultimi anni. Dapprima, si sono esposte le basi teoriche delle reti neurali che gettano le loro radici a più di mezzo secolo fa; poi si è implementato da zero un percettrone multi-strato cercando di "demistificarne" il funzionamento interno, essendo quest'ultimo poco intuitivo. Particolare enfasi è stata data alla spiegazione dell'algoritmo che è alla base dell'apprendimento delle reti, la \textbf{backpropagation of errors}. Dopodiché è stata fornita una panoramica sull'architettura delle \textbf{Convolutional Neural Networks}. Considerate lo stato dell'arte per la visione artificiale, hanno confermato le aspettative durante i test sui diversi dataset. Nel confronto fra le due architetture è emerso che avere un maggior numero di strati di convoluzione dona alla rete una maggiore potenza espressiva ed una rimarchevole capacità di astrazione e comprensione dei soggetti nelle immagini. Infine, il \textbf{transfer-learning} si è dimostrato una strategia di successo per ottenere classificatori ottimizzati per dataset arbitrari, anche piccoli. 
\\
Il framework \textbf{Torch} ha mantenuto le sue promesse sulla prototipazione flessibile e veloce. È un framework che può contare su un largo ecosistema di librerie guidate dalla comunità di ricercatori e l'astrazione che fornisce semplifica molto la progettazione e l'ottimizzazione delle reti, mantenendo al tempo stesso l'elasticità che i ricercatori necessitano per costruire modelli bleeding-edge. Tuttavia, non è facile eseguire il debugging di reti molto complesse ed, in un campo dove il trial-and-error è il pane quotidiano, sembra che ci possa essere ancora un margine di miglioramento. 

Si noti, infine, che il progetto poteva essere realizzato anche con una delle molte altre librerie per il Deep Learning (TensorFlow, Keras, Caffe, DL4J) le quali presentano ciascuna le proprie peculiarità. Vale certamente la pena seguirne gli sviluppi e sperimentarne l'utilizzo, magari con applicazioni anche in ambiti diversi. Tutto ciò dimostra quante alternative il panorama del deep learning abbia da offrire e quanto questo campo sia in crescita esponenziale.










%----------------------------------------------------------------------------------------
%	THESIS CONTENT - APPENDICES
%----------------------------------------------------------------------------------------

\appendix % Cue to tell LaTeX that the following "chapters" are Appendices

% Include the appendices of the thesis as separate files from the Appendices folder
% Uncomment the lines as you write the Appendices

%
%% Appendix A

\chapter{Appendice A: Codice addizionale ed il framework Torch} % Main appendix title

\label{AppendixA} % Change X to a consecutive letter; for referencing this appendix elsewhere, use \ref{AppendixX}

In Lua manca il costrutto delle classi. Si può tuttavia crearle utilizzando tables e meta-tables. Per il progetto si è utilizzato una piccola libreria, di seguito riportata. 

\begin{lstlisting}[language={[5.2]Lua}]
-- class.lua
-- Compatible with Lua 5.1 (not 5.0).
function class(base, init)
   local c = {} -- a new class instance
   if not init and type(base) == 'function' then
      init = base
      base = nil
   elseif type(base) == 'table' then
      -- our new class is a shallow copy of the base class!
      for i,v in pairs(base) do
         c[i] = v
      end
      c._base = base
   end
   -- the class will be the metatable for all its objects,
   -- and they will look up their methods in it.
   c.__index = c

   -- expose a constructor which can be called by <classname>(<args>)
   local mt = {}
   mt.__call = function(class_tbl, ...)
      local obj = {}
      setmetatable(obj,c)
      if init then
         init(obj,...)
      else
         -- make sure that any stuff from the base class is initialized!
         if base and base.init then
            base.init(obj, ...)
         end
      end
      return obj
   end
   c.init = init
   c.is_a = function(self, klass)
      local m = getmetatable(self)
      while m do
         if m == klass then return true end
         m = m._base
      end
      return false
   end
   setmetatable(c, mt)
   return c
end
\end{lstlisting}
% % Appendix Template

\chapter{Il framework Torch} % Main appendix title
\label{AppendixB} % Change X to a consecutive letter; for referencing this appendix elsewhere, use \ref{AppendixX}
<<<<<<< HEAD
\def \path {Appendices/}
\section{Introduzione}


\begin{lstlisting}[language={Matlab}]
%% Rankest performance evaluation 
%{
Running rankest(T) on a dense, sparse or incomplete tensor T plots an L-curve which represents the balance
between the relative error of the CPD and the number of rank-one terms R. 

The lower bound is based on the truncation error of the tensors multilinear singular values 
For incomplete and sparse tensors, this lower bound is not available and the first value to be tried for R is 1. 
The number of rank-one terms is increased until the relative error of the approximation is less than options.MinRelErr. 
In a sense, the corner of the resulting L-curve makes an optimal trade-off between accuracy and
compression. The rankest tool computes the number of rank-one terms R corresponding to
the L-curve corner and marks it on the plot with a square. This optimal number of rank-one
terms is also rankests first output.
%}

A = {}
times = []
ranks = []
n_inputs = [3, 9, 12, 32, 64]
n_outputs = [12, 32, 64, 128, 256]
kernel_size = 3; 

for i=1:5
    A{i} = abs(randn(n_inputs(i), kernel_size, kernel_size, n_outputs(i)))
    tic
    ranks(i) = rankest(A{i})
    times(i) = toc
end 

disp('estimated rankest times:') 
disp(times)
 
mul_dims = n_inputs + n_outputs; 

h3 = figure(3);
h3 = plot(mul_dims(1:end), times);
set(h1, 'LineWidth',3);
grid on

title('Rankest computation time') 
xlabel('(N. of Inputs) x (N. of Outputs)') 
ylabel('Rank estimation in SECONDS') 


%% plot
h = figure(3);
h = plot(mul_dims, times);
set(h, 'LineWidth',3);
grid on

title('Rankest computation time') 
xlabel('(N. of INPUTS) + (N. of OUTPUTS)') 
ylabel('Rank estimation in SECONDS') 

\end{lstlisting}

Torch\parencite{WTorch} è un framework per il calcolo numerico versatile che estende il linguaggio Lua. L'obiettivo è quello di fornire un ambiente flessibile per il progetto e addestramento di sistemi di machine learning anche su larga scala.\\
La flessibilità è ottenuta grazie a Lua stesso, un linguaggio di scripting estremamente leggero e efficiente. Le prestazioni sono garantite da backend compilati ed ottimizzati (C,C++,CUDA,OpenMP/SSE) per le routine di calcolo numerico di basso livello.
\\
\\
Gli obiettivi degli autori erano: (1) facilità di sviluppo di algoritmi numerici; (2) facilità di
estensione, incluso il supporto ad altre librerie; (3) la velocità.\\
Gli obiettivi (2) e (3) sono stati soddisfatti tramite l'utilizzo di Lua poiché un linguaggio interpretato risulta conveniente per il testing rapido in modo interattivo; garantisce facilità di sviluppo e, grazie alle ottime C-API, unisce in modo eterogeneo le varie librerie,
nascondendole sotto un'unica struttura di linguaggio di scripting.  Infine, essendo ossessionati dalla velocità, è stato scelto Lua poiché è un veloce linguaggio di scripting e può inoltre contare su un efficiente compilatore JIT.
Inoltre, Lua ha il grosso vantaggio di essere stato progettato per essere facilmente inserito nelle applicazioni scritte in C e consente quindi di “wrappare” le sottostanti implementazioni in C/C++ in maniera banale. Il binding C per il Lua è tra i più semplici e dona quindi grande estensibilità al progetto Torch. 
\\
\\
Torch è un framework auto-contenuto e estremamente portabile su ogni piattaforma: iOS, Android, FPGA, processori DSP ecc. Gli scrpt che vengono scritti per Torch riescono ad essere eseguiti su queste piattaforme senza nessuna modifica.
\\
Per soddisfare il requisito (1) hanno invece ideato un oggetto chiamato \texttt{'Tensor'}, il quale altro non è che una “vista” geometrica di una particolare area di memoria, e permette una efficiente e semplice 
gestione di vettori a N dimensioni, tensori appunto.\\
L’oggetto Tensor fornisce anche un’efficiente gestione della memoria: ogni operazione fatta su di esso non alloca nuova memoria, ma trasforma il tensore esistente o ritorna un nuovo tensore che referenzia la stessa area di memoria.
Torch fornisce un ricco set di routine di calcolo numerico: le comuni routine di Matlab, algebra lineare, convoluzioni, FFT, ecc. Ci sono molti package per diversi ambiti: Machine Learning, Visione Artificiale, Image \& Video Processing, Speech Recognition, ecc.\\
\\
I package più importanti per il machine learning sono: 
\begin{itemize}
\item \textbf{nn}: Neural Network, fornisce ogni sorta di modulo per la costruzione di reti neurali, reti neural profonde (deep), regressione lineare, MLP, autoencoders ecc. È il package utilizzato nel progetto. Per topologie di rete bleeding-edge si suggerisce il package \textbf{'nnx'}; 
\item  \textbf{optim}: package per l'ottimizzazione della discesa del gradiente  Fondamentale per avere buone performance nel training della rete; 
\item \textbf{unsup}: è un toolbox per l'apprendimento non supervisionato; 

\item \textbf{Image}: contiene tutte le funzioni atte all'image processing; 

\item \textbf{cunn}: package per utilizzare le reti neurali sfruttando la potenza di calcolo parallelo delle GPU, mediante l'architettura CUDA. 
\end{itemize}
L'architettura del framework è raffigurata in figura \ref{fig:torch}.\\

\begin{figure}[h!]
 \centering
 \includegraphics[width=1.0\textwidth]{\path/torch.png} 
 \caption{Architettura del framework Torch}
 \label{fig:torch}
\end{figure}
\\
Torch è adottato da una fiorente comunità attiva di ricercatori in diverse università e importanti centri di ricerca come quelli di IBM; il dipartimento di IA di Facebook (FAIR); Google DeepMind prima di passare a TensorFlow nel 2016.

\section{Utilizzo base per reti neurali}
Costruire modelli di reti neurali è una procedura rapida con Torch, ecco alcuni esempi: 
\begin{lstlisting}[language={[5.2]Lua}]
------------------------------------------------------------
-- simple linear model: logistic regression
------------------------------------------------------------
model:add(nn.Reshape(3*32*32))
model:add(nn.Linear(3*32*32,#classes))

------------------------------------------------------------
-- classic 2-layer fully-connected MLP
------------------------------------------------------------
model:add(nn.Reshape(3*32*32))
model:add(nn.Linear(3*32*32, 1*32*32))
model:add(nn.ReLU())
model:add(nn.Linear(1*32*32, #classes))

------------------------------------------------------------
-- convolutional layer
------------------------------------------------------------
--hyper-parameters
nfeats = 3 --3D input volume
nstates = {16, 64, 128} --output at each level
filtsize = 5 --filter size or kernel 
poolsize = 2

--Here's only the first stage. 
--The others look the same except for the nstates you're gonna use

-- filter bank -> squashing -> max pooling
model:add(nn.SpatialConvolutionMM(nfeats, nstates[1], filtsize, filtsize))
model:add(nn.ReLU())
model:add(nn.SpatialMaxPooling(poolsize, poolsize, poolsize, poolsize))
\end{lstlisting}
\subsection{Supporto CUDA}
CUDA (Compute Unified Device Architecture) è l'architettura di elaborazione in parallelo di NVIDIA che permette netti aumenti delle prestazioni di computing grazie allo sfruttamento della potenza di calcolo delle GPU per operazioni “general purpose”.\\
Torch offre un package chiamato \texttt{'cunn'} per usufruire di CUDA. Il package è basato su un tensore chiamato \texttt{'torch.CudaTensor()'} che altro non è che un normale Tensor che risiede ed utilizza la memoria della DRAM della GPU; tutte le operazioni definite per l'oggetto Tensor sono definite normalmente anche per il CudaTensor, il quale astrae completamente dall'utilizzo della GPU, offrendo un'interfaccia semplice e permettendo di sfruttare gli stessi script che si usano per l'elaborazione CPU. L'unica modifica da apportare, quindi, è cambiare il tipo di tensore. 
\begin{lstlisting}[language={[5.2]Lua}]
tf = torch.FloatTensor(4,100,100) -- CPU's DRAM
tc = tf:cuda() -- GPU's DRAM 
tc:mul() -- run on GPU
res = tc:float() -- res is instantiated on CPU's DRAM

--similarly, after we've built our model
--we can move it to the GPU by doing
model:cuda()
--we also need to compute our loss on GPU
criterion:cuda()

--now we're set, we can train our model on the GPU 
--just by following the standard training procedure seen in (Capitolo 4)
\end{lstlisting}


\section{ResNet}
Una rappresentazione testuale del modello a 18 strati di Residual Network. 

\begin{lstlisting}[language={[5.2]Lua}]
nn.Sequential {
  [input -> (1) -> (2) -> (3) -> (4) -> (5) -> (6) -> (7) -> (8) -> (9) -> output]
  (1): cudnn.SpatialConvolution(3 -> 16, 3x3, 1,1, 1,1) without bias
  (2): nn.SpatialBatchNormalization (4D) (16)
  (3): cudnn.ReLU
  (4): nn.Sequential {
    [input -> (1) -> (2) -> (3) -> output]
    (1): nn.Sequential {
      [input -> (1) -> (2) -> (3) -> output]
      (1): nn.ConcatTable {
        input
          |`-> (1): nn.Sequential {
          |      [input -> (1) -> (2) -> (3) -> (4) -> (5) -> output]
          |      (1): cudnn.SpatialConvolution(16 -> 16, 3x3, 1,1, 1,1) without bias
          |      (2): nn.SpatialBatchNormalization (4D) (16)
          |      (3): cudnn.ReLU
          |      (4): cudnn.SpatialConvolution(16 -> 16, 3x3, 1,1, 1,1) without bias
          |      (5): nn.SpatialBatchNormalization (4D) (16)
          |    }
           `-> (2): nn.Identity
           ... -> output
      }
      (2): nn.CAddTable
      (3): cudnn.ReLU
    }
    (2): nn.Sequential {
      [input -> (1) -> (2) -> (3) -> output]
      (1): nn.ConcatTable {
        input
          |`-> (1): nn.Sequential {
          |      [input -> (1) -> (2) -> (3) -> (4) -> (5) -> output]
          |      (1): cudnn.SpatialConvolution(16 -> 16, 3x3, 1,1, 1,1) without bias
          |      (2): nn.SpatialBatchNormalization (4D) (16)
          |      (3): cudnn.ReLU
          |      (4): cudnn.SpatialConvolution(16 -> 16, 3x3, 1,1, 1,1) without bias
          |      (5): nn.SpatialBatchNormalization (4D) (16)
          |    }
           `-> (2): nn.Identity
           ... -> output
      }
      (2): nn.CAddTable
      (3): cudnn.ReLU
    }
    (3): nn.Sequential {
      [input -> (1) -> (2) -> (3) -> output]
      (1): nn.ConcatTable {
        input
          |`-> (1): nn.Sequential {
          |      [input -> (1) -> (2) -> (3) -> (4) -> (5) -> output]
          |      (1): cudnn.SpatialConvolution(16 -> 16, 3x3, 1,1, 1,1) without bias
          |      (2): nn.SpatialBatchNormalization (4D) (16)
          |      (3): cudnn.ReLU
          |      (4): cudnn.SpatialConvolution(16 -> 16, 3x3, 1,1, 1,1) without bias
          |      (5): nn.SpatialBatchNormalization (4D) (16)
          |    }
           `-> (2): nn.Identity
           ... -> output
      }
      (2): nn.CAddTable
      (3): cudnn.ReLU
    }
  }
  (5): nn.Sequential {
    [input -> (1) -> (2) -> (3) -> output]
    (1): nn.Sequential {
      [input -> (1) -> (2) -> (3) -> output]
      (1): nn.ConcatTable {
        input
          |`-> (1): nn.Sequential {
          |      [input -> (1) -> (2) -> (3) -> (4) -> (5) -> output]
          |      (1): cudnn.SpatialConvolution(16 -> 32, 3x3, 2,2, 1,1) without bias
          |      (2): nn.SpatialBatchNormalization (4D) (32)
          |      (3): cudnn.ReLU
          |      (4): cudnn.SpatialConvolution(32 -> 32, 3x3, 1,1, 1,1) without bias
          |      (5): nn.SpatialBatchNormalization (4D) (32)
          |    }
           `-> (2): nn.Sequential {
                 [input -> (1) -> (2) -> output]
                 (1): nn.SpatialAveragePooling(1x1, 2,2)
                 (2): nn.Concat {
                   input
                     |`-> (1): nn.Identity
                      `-> (2): nn.MulConstant
                      ... -> output
                 }
               }
           ... -> output
      }
      (2): nn.CAddTable
      (3): cudnn.ReLU
    }
    (2): nn.Sequential {
      [input -> (1) -> (2) -> (3) -> output]
      (1): nn.ConcatTable {
        input
          |`-> (1): nn.Sequential {
          |      [input -> (1) -> (2) -> (3) -> (4) -> (5) -> output]
          |      (1): cudnn.SpatialConvolution(32 -> 32, 3x3, 1,1, 1,1) without bias
          |      (2): nn.SpatialBatchNormalization (4D) (32)
          |      (3): cudnn.ReLU
          |      (4): cudnn.SpatialConvolution(32 -> 32, 3x3, 1,1, 1,1) without bias
          |      (5): nn.SpatialBatchNormalization (4D) (32)
          |    }
           `-> (2): nn.Identity
           ... -> output
      }
      (2): nn.CAddTable
      (3): cudnn.ReLU
    }
    (3): nn.Sequential {
      [input -> (1) -> (2) -> (3) -> output]
      (1): nn.ConcatTable {
        input
          |`-> (1): nn.Sequential {
          |      [input -> (1) -> (2) -> (3) -> (4) -> (5) -> output]
          |      (1): cudnn.SpatialConvolution(32 -> 32, 3x3, 1,1, 1,1) without bias
          |      (2): nn.SpatialBatchNormalization (4D) (32)
          |      (3): cudnn.ReLU
          |      (4): cudnn.SpatialConvolution(32 -> 32, 3x3, 1,1, 1,1) without bias
          |      (5): nn.SpatialBatchNormalization (4D) (32)
          |    }
           `-> (2): nn.Identity
           ... -> output
      }
      (2): nn.CAddTable
      (3): cudnn.ReLU
    }
  }
  (6): nn.Sequential {
    [input -> (1) -> (2) -> (3) -> output]
    (1): nn.Sequential {
      [input -> (1) -> (2) -> (3) -> output]
      (1): nn.ConcatTable {
        input
          |`-> (1): nn.Sequential {
          |      [input -> (1) -> (2) -> (3) -> (4) -> (5) -> output]
          |      (1): cudnn.SpatialConvolution(32 -> 64, 3x3, 2,2, 1,1) without bias
          |      (2): nn.SpatialBatchNormalization (4D) (64)
          |      (3): cudnn.ReLU
          |      (4): cudnn.SpatialConvolution(64 -> 64, 3x3, 1,1, 1,1) without bias
          |      (5): nn.SpatialBatchNormalization (4D) (64)
          |    }
           `-> (2): nn.Sequential {
                 [input -> (1) -> (2) -> output]
                 (1): nn.SpatialAveragePooling(1x1, 2,2)
                 (2): nn.Concat {
                   input
                     |`-> (1): nn.Identity
                      `-> (2): nn.MulConstant
                      ... -> output
                 }
               }
           ... -> output
      }
      (2): nn.CAddTable
      (3): cudnn.ReLU
    }
    (2): nn.Sequential {
      [input -> (1) -> (2) -> (3) -> output]
      (1): nn.ConcatTable {
        input
          |`-> (1): nn.Sequential {
          |      [input -> (1) -> (2) -> (3) -> (4) -> (5) -> output]
          |      (1): cudnn.SpatialConvolution(64 -> 64, 3x3, 1,1, 1,1) without bias
          |      (2): nn.SpatialBatchNormalization (4D) (64)
          |      (3): cudnn.ReLU
          |      (4): cudnn.SpatialConvolution(64 -> 64, 3x3, 1,1, 1,1) without bias
          |      (5): nn.SpatialBatchNormalization (4D) (64)
          |    }
           `-> (2): nn.Identity
           ... -> output
      }
      (2): nn.CAddTable
      (3): cudnn.ReLU
    }
    (3): nn.Sequential {
      [input -> (1) -> (2) -> (3) -> output]
      (1): nn.ConcatTable {
        input
          |`-> (1): nn.Sequential {
          |      [input -> (1) -> (2) -> (3) -> (4) -> (5) -> output]
          |      (1): cudnn.SpatialConvolution(64 -> 64, 3x3, 1,1, 1,1) without bias
          |      (2): nn.SpatialBatchNormalization (4D) (64)
          |      (3): cudnn.ReLU
          |      (4): cudnn.SpatialConvolution(64 -> 64, 3x3, 1,1, 1,1) without bias
          |      (5): nn.SpatialBatchNormalization (4D) (64)
          |    }
           `-> (2): nn.Identity
           ... -> output
      }
      (2): nn.CAddTable
      (3): cudnn.ReLU
    }
  }
  (7): cudnn.SpatialAveragePooling(8x8, 1,1)
  (8): nn.View(64)
  (9): nn.Linear(64 -> 10)
}

=======
\section{Introduzione}
\parencite{WTorch}

\section{Utilizzo base per reti neurali}

\subsection{Supporto CUDA}

\section{ResNet}
Una rappresentazione testuale del modello a 18 strati di Residual Network. 

\begin{lstlisting}[language={[5.2]Lua}]
nn.Sequential {
  [input -> (1) -> (2) -> (3) -> (4) -> (5) -> (6) -> (7) -> (8) -> (9) -> output]
  (1): cudnn.SpatialConvolution(3 -> 16, 3x3, 1,1, 1,1) without bias
  (2): nn.SpatialBatchNormalization (4D) (16)
  (3): cudnn.ReLU
  (4): nn.Sequential {
    [input -> (1) -> (2) -> (3) -> output]
    (1): nn.Sequential {
      [input -> (1) -> (2) -> (3) -> output]
      (1): nn.ConcatTable {
        input
          |`-> (1): nn.Sequential {
          |      [input -> (1) -> (2) -> (3) -> (4) -> (5) -> output]
          |      (1): cudnn.SpatialConvolution(16 -> 16, 3x3, 1,1, 1,1) without bias
          |      (2): nn.SpatialBatchNormalization (4D) (16)
          |      (3): cudnn.ReLU
          |      (4): cudnn.SpatialConvolution(16 -> 16, 3x3, 1,1, 1,1) without bias
          |      (5): nn.SpatialBatchNormalization (4D) (16)
          |    }
           `-> (2): nn.Identity
           ... -> output
      }
      (2): nn.CAddTable
      (3): cudnn.ReLU
    }
    (2): nn.Sequential {
      [input -> (1) -> (2) -> (3) -> output]
      (1): nn.ConcatTable {
        input
          |`-> (1): nn.Sequential {
          |      [input -> (1) -> (2) -> (3) -> (4) -> (5) -> output]
          |      (1): cudnn.SpatialConvolution(16 -> 16, 3x3, 1,1, 1,1) without bias
          |      (2): nn.SpatialBatchNormalization (4D) (16)
          |      (3): cudnn.ReLU
          |      (4): cudnn.SpatialConvolution(16 -> 16, 3x3, 1,1, 1,1) without bias
          |      (5): nn.SpatialBatchNormalization (4D) (16)
          |    }
           `-> (2): nn.Identity
           ... -> output
      }
      (2): nn.CAddTable
      (3): cudnn.ReLU
    }
    (3): nn.Sequential {
      [input -> (1) -> (2) -> (3) -> output]
      (1): nn.ConcatTable {
        input
          |`-> (1): nn.Sequential {
          |      [input -> (1) -> (2) -> (3) -> (4) -> (5) -> output]
          |      (1): cudnn.SpatialConvolution(16 -> 16, 3x3, 1,1, 1,1) without bias
          |      (2): nn.SpatialBatchNormalization (4D) (16)
          |      (3): cudnn.ReLU
          |      (4): cudnn.SpatialConvolution(16 -> 16, 3x3, 1,1, 1,1) without bias
          |      (5): nn.SpatialBatchNormalization (4D) (16)
          |    }
           `-> (2): nn.Identity
           ... -> output
      }
      (2): nn.CAddTable
      (3): cudnn.ReLU
    }
  }
  (5): nn.Sequential {
    [input -> (1) -> (2) -> (3) -> output]
    (1): nn.Sequential {
      [input -> (1) -> (2) -> (3) -> output]
      (1): nn.ConcatTable {
        input
          |`-> (1): nn.Sequential {
          |      [input -> (1) -> (2) -> (3) -> (4) -> (5) -> output]
          |      (1): cudnn.SpatialConvolution(16 -> 32, 3x3, 2,2, 1,1) without bias
          |      (2): nn.SpatialBatchNormalization (4D) (32)
          |      (3): cudnn.ReLU
          |      (4): cudnn.SpatialConvolution(32 -> 32, 3x3, 1,1, 1,1) without bias
          |      (5): nn.SpatialBatchNormalization (4D) (32)
          |    }
           `-> (2): nn.Sequential {
                 [input -> (1) -> (2) -> output]
                 (1): nn.SpatialAveragePooling(1x1, 2,2)
                 (2): nn.Concat {
                   input
                     |`-> (1): nn.Identity
                      `-> (2): nn.MulConstant
                      ... -> output
                 }
               }
           ... -> output
      }
      (2): nn.CAddTable
      (3): cudnn.ReLU
    }
    (2): nn.Sequential {
      [input -> (1) -> (2) -> (3) -> output]
      (1): nn.ConcatTable {
        input
          |`-> (1): nn.Sequential {
          |      [input -> (1) -> (2) -> (3) -> (4) -> (5) -> output]
          |      (1): cudnn.SpatialConvolution(32 -> 32, 3x3, 1,1, 1,1) without bias
          |      (2): nn.SpatialBatchNormalization (4D) (32)
          |      (3): cudnn.ReLU
          |      (4): cudnn.SpatialConvolution(32 -> 32, 3x3, 1,1, 1,1) without bias
          |      (5): nn.SpatialBatchNormalization (4D) (32)
          |    }
           `-> (2): nn.Identity
           ... -> output
      }
      (2): nn.CAddTable
      (3): cudnn.ReLU
    }
    (3): nn.Sequential {
      [input -> (1) -> (2) -> (3) -> output]
      (1): nn.ConcatTable {
        input
          |`-> (1): nn.Sequential {
          |      [input -> (1) -> (2) -> (3) -> (4) -> (5) -> output]
          |      (1): cudnn.SpatialConvolution(32 -> 32, 3x3, 1,1, 1,1) without bias
          |      (2): nn.SpatialBatchNormalization (4D) (32)
          |      (3): cudnn.ReLU
          |      (4): cudnn.SpatialConvolution(32 -> 32, 3x3, 1,1, 1,1) without bias
          |      (5): nn.SpatialBatchNormalization (4D) (32)
          |    }
           `-> (2): nn.Identity
           ... -> output
      }
      (2): nn.CAddTable
      (3): cudnn.ReLU
    }
  }
  (6): nn.Sequential {
    [input -> (1) -> (2) -> (3) -> output]
    (1): nn.Sequential {
      [input -> (1) -> (2) -> (3) -> output]
      (1): nn.ConcatTable {
        input
          |`-> (1): nn.Sequential {
          |      [input -> (1) -> (2) -> (3) -> (4) -> (5) -> output]
          |      (1): cudnn.SpatialConvolution(32 -> 64, 3x3, 2,2, 1,1) without bias
          |      (2): nn.SpatialBatchNormalization (4D) (64)
          |      (3): cudnn.ReLU
          |      (4): cudnn.SpatialConvolution(64 -> 64, 3x3, 1,1, 1,1) without bias
          |      (5): nn.SpatialBatchNormalization (4D) (64)
          |    }
           `-> (2): nn.Sequential {
                 [input -> (1) -> (2) -> output]
                 (1): nn.SpatialAveragePooling(1x1, 2,2)
                 (2): nn.Concat {
                   input
                     |`-> (1): nn.Identity
                      `-> (2): nn.MulConstant
                      ... -> output
                 }
               }
           ... -> output
      }
      (2): nn.CAddTable
      (3): cudnn.ReLU
    }
    (2): nn.Sequential {
      [input -> (1) -> (2) -> (3) -> output]
      (1): nn.ConcatTable {
        input
          |`-> (1): nn.Sequential {
          |      [input -> (1) -> (2) -> (3) -> (4) -> (5) -> output]
          |      (1): cudnn.SpatialConvolution(64 -> 64, 3x3, 1,1, 1,1) without bias
          |      (2): nn.SpatialBatchNormalization (4D) (64)
          |      (3): cudnn.ReLU
          |      (4): cudnn.SpatialConvolution(64 -> 64, 3x3, 1,1, 1,1) without bias
          |      (5): nn.SpatialBatchNormalization (4D) (64)
          |    }
           `-> (2): nn.Identity
           ... -> output
      }
      (2): nn.CAddTable
      (3): cudnn.ReLU
    }
    (3): nn.Sequential {
      [input -> (1) -> (2) -> (3) -> output]
      (1): nn.ConcatTable {
        input
          |`-> (1): nn.Sequential {
          |      [input -> (1) -> (2) -> (3) -> (4) -> (5) -> output]
          |      (1): cudnn.SpatialConvolution(64 -> 64, 3x3, 1,1, 1,1) without bias
          |      (2): nn.SpatialBatchNormalization (4D) (64)
          |      (3): cudnn.ReLU
          |      (4): cudnn.SpatialConvolution(64 -> 64, 3x3, 1,1, 1,1) without bias
          |      (5): nn.SpatialBatchNormalization (4D) (64)
          |    }
           `-> (2): nn.Identity
           ... -> output
      }
      (2): nn.CAddTable
      (3): cudnn.ReLU
    }
  }
  (7): cudnn.SpatialAveragePooling(8x8, 1,1)
  (8): nn.View(64)
  (9): nn.Linear(64 -> 10)
}

>>>>>>> ece01a6c88b2ea9153728cdbf63dcf2c83a18f6a
\end{lstlisting}
%\include{Appendices/AppendixC}

%----------------------------------------------------------------------------------------
%	BIBLIOGRAPHY
%----------------------------------------------------------------------------------------

\printbibliography[heading=bibintoc]

%----------------------------------------------------------------------------------------

\end{document}  
